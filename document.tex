\documentclass[11pt,letterpaper]{book}
\usepackage{emptypage}
\usepackage{pdflscape}
% math stuffs
\usepackage{mathtools}
%\usepackage{wasysym}
\usepackage{seqsplit}
\usepackage{multirow}

% fontenc for the accents and junk
\usepackage[utf8]{luainputenc}
\usepackage[T1]{fontenc}

% Typeface
%\usepackage[adobe-utopia]{mathdesign}
%\usepackage{fourier}
%\usepackage{fouriernc}
%\usepackage{mathptmx}
%\usepackage{tgtermes}
%\usepackage{pxfonts}
%\usepackage{newpxtext,newpxmath}
%\usepackage[sc]{mathpazo}
%\usepackage[oldstylenums]{kpfonts}
\usepackage{kpfonts}
%\usepackage{tgpagella}
%\usepackage[garamond]{mathdesign}
%\usepackage{garamondx}
%\usepackage{ebgaramond}

% Need Greek character support
\usepackage{textgreek}

% Umm
\usepackage{adjustbox}
\usepackage{multirow}
\usepackage{verbatimbox}
\usepackage{caption}
\usepackage{ftnxtra}
\usepackage{longtable}
\usepackage[rightmargin=0.0pt,indentfirst=false]{quoting}

% Makes graphics easy
\usepackage{graphicx}

\usepackage{titling}

% PDF is always pretty
\DeclareGraphicsExtensions{.pdf}

% Line spacing, 1.4 is pretty but YMMV
\linespread{1.4}

% Homespun macros 
%\newcommand{\wlt}[1]{{\Large\;\textless\normalsize}#1{\Large\textgreater\;\normalsize}}
%\newcommand{\wlc}[1]{{\large\;\{\normalsize}#1{\large\}\;\normalsize}}
%\renewcommand{\arraystretch}{0.7}
\newcommand{\Hair}{\ifmmode\mskip1mu\else\kern0.12em\fi}

\begin{document}

\frontmatter

\setlength{\droptitle}{-96pt}

\pretitle{\begin{flushleft}}
\title{\Huge{\textsc{The True Squaring of the Circle and the Hyperbola}} \\
\vskip 1.0em \Large{On its own kind of proportion, \\
Discovered and proved.}}
\posttitle{\par\end{flushleft}\vskip 4.5em}

\preauthor{\begin{flushleft}}
\author{\LARGE{James Gregory} \vskip 11.0em}
\postauthor{\end{flushleft}}

\predate{\begin{tabular}{p{140pt} p{100pt}} &\raggedleft}
\date{1667 \\ Aberdeen, Scottland}
\postdate{\end{tabular}}

\maketitle

\chapter*{\Huge\textnormal{\textsc{Welcome,}} \\ \textnormal{\textsc{Reader
of Geometry}}}

I was contemplating at length, dear Reader, whether analysis with its many
operations, along with the general method of investigating all proportions of a
quantity, are adequate, as Descartes first affirms in his Geometry. Indeed
if it is so, it may be that it is possible to use it to demonstrate the
oft-sung feat of squaring of the circle. Whenever I turn this over in my mind I
readily see from the discoveries made so far on the properties of the circle
that analysis may not suffice to establish such results. It occurred to me that
I should only research the more general case after first considering the common
circle. In doing so I hit upon the convergent sequence of polygons, the limit of
which is the circular sector, and I saw here at once a trace of analysis. It
followed that the convergent series applies naturally not only in the case of
the circle, but likewise in more general consideration. And so from the
properties of the circle followed the cases of ellipse and the hyperbola with no
effort, and thus the infallible quadrature of every conic section was revealed
to me. However the convergent series of polygons soon turned the tables on me,
and I was forced to ply all of my arts to the insuperable difficulty of
discovering its limit. But then I remembered that the spirit of analysis is the
same as that of the common algebra, not only in resolving problems, but likewise
in demonstrating their impossibility (when it be useful). After this first
difficulty was put to rest, I also conquered the more general case, as I claim
above. Indeed I uncovered the true squaring, by its own manner of proportion,
not only of the circle (which I had set out to do at the beginning) but of all
of the conic sections, and I revealed an entirely new type of proportion
previously unknown in the sphere of Geometry. I can draw out this proportion as
closely as needed in relation to the dimension of the conic section, and can
easily demonstrate the very rapid extraction of roots of solid surds (unless I
am very much mistaken). In fact, such approximations apply mathematically to
every incommensurable proportion, such that we might better understand the
nature of the proportion. I might speak of a proportion insofar as it is
derived by some analytic operations (i.e. of the common arithmetic), and the
proportion is then known to us in numbers or in the continuation of some
discrete quantities, and I am not afraid to render these operations
geometrically in the style of Decartes.

First note that we always know how to relate commensurable quantities, or
those having a common number between them. We cannot perceive incommensurable
proportions unless they themselves are relatively commensurable, for the
infinitude renders us ignorant, blunting our minds and impeding our simple
notions. Two of the five arithmetic operations, addition and subtraction, are
the most elementary. Multiplication is composed of addition, division from
subtraction. Root extraction, which in general is nothing more than finding
commensurable proportions, and which closely approaches an analytically
incommensurable proportion, is composed from the aforementioned four. Our sixth
operation, which in general is nothing more than finding commensurable
proportions closely approximating a non-analytic proportion, is composed of the
previous five. It will be noted that just as no fractional numbers are produced
from integers by addition, subtraction or multiplication, but only by division,
and just as no incommensurable numbers are produced from commensurables by
addition subtraction, multiplication or division, but only by root extraction,
so non-analytic quantities are not born from analytic numbers by addition,
subtraction, multiplication, division, or root extraction, but rather from this
sixth operation. As such this new operation of ours may be added to arithmetic
and a new type of ratio may be added to geometry, which (as I demonstrate in
this treatise) allows us to understand the analyticity of the ratio of the
circle to the square of its diameter as a type of ratio we currently know how to
exhibit, as root extraction allows us to understand the ratio between the side
of the square and its diameter in terms of commensurability.

It seems clear to me that the whole of the proof cannot be reduced to
gemometric language, for in order to accomplish this we would use no small
number of analytic magnitudes that are mutually related but generally
incommensurable, which I am amazed have never been written about, as they open
wide the plains of discovery.
For from this demonstration is to be attacked, which by the power of the
mesolabe cannot be accomplished by the straightedge and compass, likewise which
not always and where the impaired equations can be reduced to purity, likewise
which be necessary at minimum such general curve to mechanic resolution of the
equation with so great innumerable, which from the existing geometry be
impossible to be discerned from analysis, and from the undeveloped ones every
day and in vain are sought.
Euclid dedicated the entirety of his tenth book (except for a few minor general
propositions) on incommensurables produced from theextraction of square roots.
And I don't know how many by any other treatise is this material, although
it be not only very useful to speculative geometry but is also an extremely
remarkable theory in its own right.
For example, in a geometric progression in which the first term is given by a
commensurable length or by a power of such a quantity, and the second is any
other whichever, be it binomial, trinomial, etc., it is impossible that the
sequences progressing infinitely from these two terms will be mutually
commensurable by a length or a power of such.
Another much more to report, but perhaps i reserve this for a more favorable
time, it is enough considering for the present this to have proved analytically.
And if indeed analysis agreeing to such a violent degree does not compel and
geometry, never repelling nore being repelled (by?) geometry, which analysis
once proved geometric.
From this invention I deduce some new angular sections and a doctrine of
logarithms, some easy, most expeditious in practice and fortified by geometric
demonstration.
As far as the construction of the logarithm very long, the known conjecture is
preferably seen, and division of the angle into more than five equal parts.
Counting by prime numbers can be done only with great difficulty.
I demonstrate all of these sums (which I can) quickly and clearly.
And I am not careful in citations, in as much as foreign books to such work are
destitute, and indeed I suppose you are not mediocrely versed in geometry,
otherwise you will acquire no fruit by this.
It remains as I may remind you the sum always to aspire to demonstration of
generality, not only in this but also in other things produced by me.
Thus as I admire R. P. Fracis Eschinard from the Society of Jesus in his optical
dialog on page 26 to confirm to have proved to me in my promota optica, that
image is seen in the same angle from the vertex of emersion as visible from the
incident vertex, not only in elliptic and hyperbolic lenses, but also in all of
my particular theory from which the whole of optical science is deduced, namely
pages 26-27, 44-47, and 50.

\ldots

\mainmatter
\begin{center}
\Huge\textsc{Definitiones.}
\end{center}
\begin{enumerate}
  \item Let two lines be drawn from the center of a circle, ellipse, or
  hyperbola to its perimeter. We call the piece bound by those two lines and
  the segment of the perimeter a sector.
  
  \item Let the segment of the perimeter between the two lines be subtended
  any number of times, forming rectilinear triangles (where the common vertex
  is the center of the conic section and the bases are the subtending lines).
  Then if the conic section is a circle or ellipse we call the figure created
  by combining these triangles a regular inscribed triangle, if it is a
  hyperbola then we call the figure a regular circumscribed polygon.
  
  \item Let the segment of the perimeter between the lines be made tangent to
  lines any number of times and let lines be drawn from the tangents to the
  center of the conic section, and further require that each of the
  quadrilaterals, understood as being composed from successive tangent lines
  and the lines to the center, be equal. If the conic section is a circle or an
  ellipse, then I call the figure created by the combination of these a regular
  circumscribed polygon. If it is a hyperbola then I call it a regular
  inscribed polygon.
  
  \item Let all of the vertices of the subtending angles (except those to the
  center of the conic section) of the regular polygon tough all of the points
  of contact of the regular polygon with the tangents. I call this a complex
  polygon.
  
  \item We say a magnitude is composed of magnitudes, when one magnitude
  makes other magnitudes by addition, subtraction, multiplication, division,
  root extraction, or any other operation imaginable.
  
  \item When a magnitude is composed of magnitudes by addition, subtraction,
  multiplication, division, or root extraction, we say it is composed
  analytically.
  
  \item When magnitudes can be composed analytically of magnitudes which are
  mutually commensurable, we say they are mutually analytic.
  
  \item Let the magnitude $X$ be composed of some magnitudes $A$, $B$, $C$, $D$,
  and $E$, and the magnitude $Z$ be composed of magnitudes $F$, $G$, $C$, $D$,
  and $E$ by the same method and operations as $X$ only with the magnitudes $F$
  and $G$ in place of $A$ and $B$. If this is so, then we say that the magnitude
  $X$ is composed of $A$ and $B$ by the same method as $Z$ is composed of $F$
  and $G$.
  
  \item Let there be two magnitudes $A$ and $B$ of which are composed two other
  magnitudes $C$ and $D$, whose the difference is less than that of $A$ and $B$.
  Also let $E$ be composed from $C$ and $D$ by the same method as $C$ is
  composed of $A$ and $B$, and $F$ be composed from $C$ and $D$ by the same
  method as $D$ is composed of $A$ and $B$. And further still let $G$ be
  composed from $E$ and $F$ by the same method as $E$ is composed of $C$ and $D$
  and $C$ is composed by $A$ and $B$, and let $H$ be composed from $E$ and $F$
  by the same method as $F$ is composed of $C$ and $D$ and $C$ is composed by
  $A$ and $B$. Continue thus. I call this a convergent series.
  
  \item Terms being placed next to each other as $A$ and $B$, or $C$ and $D$,
  or $E$ and $F$, or $G$ and $H$, are called convergent terms.
\end{enumerate}

\begin{center}
\Large\textsc{Petitions.}
\end{center}
\begin{enumerate}
  \item We desire that magnitudes composed of given mutually analytic
  magnitudes be mutually analytic themselves as well as analytic with the given
  magnitudes.
  
  \item Likewise we desire that magnitudes that cannot be analytically
  composed from given mutually analytic magnitudes not be analytic with the
  given magnitudes.
\end{enumerate}
The preceding desires may perhaps be seen by some as obscure, but will be made
clear by an analysis of the elements.

\cleardoublepage
\begin{center}
\Huge\textsc{The True Squaring of the Circle and the Hyperbola.}
\end{center}

\vskip 2.0em
Let $BIP$ be a segment of a circle, ellipse, or hyperbola with center $A$. The
triangle $ABP$ may be completed, and from the points $B$ and $P$ on the segment
tangents $BF$ and $PF$ may be drawn, which will meet each other at the point
$F$. The line $AF$ is thus produced, which intersects the segment at point $I$
and the line $BP$ at the point $Q$. From this the lines $BI$ and $PI$ are
joined.

\begin{figure}[h]
\includegraphics[scale=0.95]{vera_i}
\end{figure}

\newpage
\begin{samepage}
\begin{center}
\large\textsc{Proposition I. Theorem.}

\vskip 1.0em
\emph{The quadrilateral $BAPI$ is half of the quadrilateral $BAPF$ plus the
triangle $BAP$.}
\end{center}
\end{samepage}

The line $AQ$ is drawn through $F$ meeting with the the two lines $FB$ and $FP$,
which are tangent to the segment at the points $B$ and $P$. Therefore the line
$AQ$ contacts the line $BP$ at its bisector $Q$. We see from this that the triangle
$AQB$ is equal to the triangle $AQP$, the triangle $FBQ$ equals triangle
$FQP$, and the triangle $ABF$ equals triangle $APF$. Therefore the triangle
$ABF$ is half of the quadrilateral $ABFP$. Similarly the triangle $ABI$ is half
of the quadrilateral $ABIP$ and the triangle $ABQ$ is half of triangle $ABP$.
$ABF$, $ABI$, and $ABQ$ each have the same altitude and share one base, but the
other bases $AF$, $AI$, and $AQ$ progress arithmetically. Therefore the two
quadrilaterals $ABFP$ and $ABIP$ and the triangle $ABP$ are clearly in
arithmetic progression with a ratio of $AF$ to $AI$, Q.E.D.

\vskip 1.0em
Let the line $DL$ be drawn tangent to the segment at the point $I$ so that it
will meet the lines $BF$ and $PF$ at the points $D$ and $L$, completing the
polygon $ABDLP$.

\begin{samepage}
\begin{center}
\large\textsc{Proposition II. Theorem.}

\vskip 1.0em
\emph{The quadrilateral $ABFP$ plus quadrilateral $ABIP$ is to the double of
quadrilateral $ABIP$ as quadrilateral $ABFP$ is to polygon $ABDLP$.}
\end{center}
\end{samepage}

The line $AF$ is drawn through the point of tangency of the line $DL$ with the
segment, and is likewise drawn through the meeting point of the two lines $FB$
and $FP$, which terminate the line $DL$ and touch the segment in two points.
Therefore the line $DL$ is bisected at the point $I$. Because of this the
triangle $FDI$ equals the triangle $FIL$ and the triangle $ABF$ equals the
triangle $APF$. Thus the quadrilateral $ABDI$ equals quadrilateral $APLI$, and
further quadrilateral $APLI$ is half of the polygon $ABDLP$. It is obvious from
the above demonstration that the triangle $AIL$ equals triangle $ALP$, but
triangle $ALF$ is to triangle $ALI$ as $FA$ is to $AI$ and $FA$ is to $AI$ as
quadrilateral $ABPF$ is to quadrilateral $ABIP$. Thus quadrilateral $ABFP$ is to
quadrilateral $ABIP$ as triangle $ALF$ is to triangle $AIL$. So putting it
together, quadrilateral $ABFP$ plus $ABIP$ is to quadrilateral $ABIP$ as
triangle $AFL$ plus triangle $AIL$, i.e. triangle $AFP$, is to triangle $AIL$.
Doubling this result, $ABFP$ plus $ABIP$ is to the double of $ABIP$ as triangle
$AFP$ is to quadrilateral $AILP$. Note that triangle $AFP$ is half of
quadrilateral $ABFP$ and quadrilateral $AILP$ is half of polygon $ABDLP$.
Therefore quadrilateral $ABFP$ plus quadrilateral $ABIP$ is to the double of
$ABIP$ as quadrilateral $ABFP$ is to polygon $ABDLP$, Q.E.D.

\begin{samepage}
\begin{center}
\large\textsc{Proposition III. Theorem.}

\vskip 1.0em
\emph{The triangle $BAP$ plus the quadrilateral $ABIP$ is to the quadrilateral
$ABIP$ as the double of the quadrilateral $ABIP$ is to the polygon $ABDLP$.}
\end{center}
\end{samepage}

In the preceding proposition it is shown that the sum of $ABFP$ and $ABIP$ is to
the double of quadrilateral $ABIP$ as quadrilateral $ABFP$ is to polygon
$ABDLP$. By permuting we see that quadrilateral $ABFP$ plus $ABIP$ is to
quadrilateral $ABFP$ as the double of quadrilateral $ABIP$ is to polygon
$ABDLP$. Since quadrilaterals $ABFP$ and $ABIP$ and triangle $ABP$ are in
arithmetic progression, we have that quadrilateral $ABIP$ is to quadrilateral
$ABFP$ as triangle $ABP$ is to quadrilateral $ABIP$. Putting these together,
quadrilateral $ABIP$ plus $ABFP$ is to quadrilateral $ABFP$ as triangle $ABP$
plus quadrilateral $ABIP$ is to quadrilateral $ABIP$. On the other hand $ABIP$
plus $ABFP$ is to quadrilateral $ABFP$ as the double of quadrilateral $ABIP$ is
to polygon $ABDLP$. And therefore triangle $ABP$ plus quadrilateral $ABIP$ is
to quadrilateral $ABIP$ as the double of quadrilateral $ABIP$ is to polygon
$ABDLP$, Q.E.D.

\vskip 1.0em
Let the lines $AD$ and $AL$ be drawn to meet the segment at the points $E$
and $O$ and intersecting the lines $BI$ and $IP$ at $H$ and $M$. From this are
joined the lines $BE$, $EI$, $IO$, and $OP$ to complete the polygon $ABEIOP$.

\begin{samepage}
\begin{center}
\large\textsc{Proposition IV. Theorem.}

\vskip 1.0em
\emph{The polygon $ABEIOP$ is half of $ABDLP$ plus the quadrilateral $ABIP$.}
\end{center}
\end{samepage}

From the previous theorem it is obvious that the quadrilateral $AILP$,
quadrilateral $AIOP$, and triangle $AIP$ are in arithmetic progression, and from
the previous proposition one can gather easily enough that the quadrilateral
$AILP$ is half of polygon $ABDLP$, quadrilateral $AIOP$ is half of polygon
$ABEIOP$, and triangle $AIP$ is half of quadrilateral $ABIP$. Thus by doubling
the terms, polygon $ABDLP$, polygon $ABEIOP$, and quadrilateral $ABIP$ are in
arithmetic progression, Q.E.D.

\vskip 1.0em
Let lines $CG$ and $KN$ be drawn tangent to the segment at the points $E$ and
$O$ and let the lines $DL$, $DB$, and $LP$ intersect at the points $C$, $G$,
$K$, and $N$ to complete the polygon $ABCGKNP$.

\begin{samepage}
\begin{center}
\large\textsc{Proposition V. Theorem.}

\vskip 1.0em
\emph{The quadrilateral $ABIP$ plus the polygon $ABEIOP$ are to the polygon
$ABEIOP$ as the double of the polygon $ABEIOP$ is to the polygon $ABCGKNP$.}
\end{center}
\end{samepage}

From the third theorem it is obvious that the triangle $ABI$ plus the
quadrilateral $ABEI$ is to the quadrilateral $ABEI$ as the double of
quadrilateral $ABEI$ is to polygon $ABCGI$. From the previous proposition it is
easily concluded that triangle $ABI$ is half of quadrilateral $ABIP$,
quadrilateral $ABEI$ is half of polygon $ABEIOP$, and polygon $ABCGI$ is half of
polygon $ABCGKNP$. Therefore by doubling the terms, quadrilateral $ABIP$ plus
$ABEIOP$ is to polygon $ABEIOP$ as the double of polygon $ABEIOP$ is to polygon
$ABCGKNP$, Q.E.D.

\vskip 1.0em
From this one can easily see that the polygon $ABCGKNP$ is the harmonic mean
between polygons $ABEIOP$ and $ABDLP$, which is sufficient to suggest that this
may be demonstrated perpetually.

\begin{samepage}
\begin{center}
\large\textsc{Scholium.}
\end{center}
\end{samepage}

The two preceding propositions can be proved by the same method for whichever
complex polygons in place of the complex polygons $ABIP$ and $ABDLP$. Indeed the
tangent polygon contains as many equal quadrilaterals as the subtending polygon
contains triangles. And so it is evident that these ratios of the polygons
continue themselves to infinity, drawing lines $AN$, $AK$, $AG$, and $AC$
through points $R$, $T$, $S$, and $V$ and composing other lines and polygons
inside and outside of these. Note that we may say of these inscribed and
circumscribed polygons that they double by inscription and circumscription.

From the previous propositions it is obvious (if we let triangle $ABP = a$ and
quadrilateral $ABFP = b$) that quadrilateral $ABIP = \sqrt{ab}$ and polygon
$ABDLP = \frac{2ab}{a+\sqrt{ab}}$. By the same method, let quadrilateral $ABIP =
c$ and polygon $ABDLP = d$ and we have that polygon $ABEIOP = \sqrt{cd}$ and
polygon $ABCGKNP = \frac{2cd}{c+\sqrt{cd}}$, and it is evindent from this that
the series of polygons converges.

And so, continuing this to infinity, it is obvious that in the end we have shown
that the magnitude of the sector of the circle, the ellipse, or the hyperbola
equals $ABEIOP$. Indeed the difference of the complex polygons in the series
always diminishes, so that all of the magnitudes may be made smaller, and so in
the following theorems we shall demonstrate the Scholium.

Therefore if the aforementioned series of polygons terminates, that is,
if one may find a final inscribed polygon (if we may call it that) equal to the
final circumscribed polygon, one would infallibly have the quadrature of the
circle and the hyperbola. But since it has proved difficult, and in geometry it
is perhaps altogether impossible for such a series to terminate, certain
propositions are permitted from which to find this kind of limit of the series.
And eventually (if it is possible) the general method for finding all of the
limits of convergent series.

\begin{samepage}
\begin{center}
\large\textsc{Proposition VI. Theorem.}

\vskip 1.0em
\emph{The difference between the triangle $ABP$ and the quadrilateral $ABPF$
is greater than twice the difference between the quadrilateral $ABIP$ and the
polygon $ABDLP$.}
\end{center}
\end{samepage}

Denote the triangle $ABP$ by $A$, quadrilateral $ABFP$ by $B$, quadrilateral $ABIP$ by
$C$, and polygon $ABDLP$ by $D$. Since $A$ is to $C$ as $C$ is to $B$, the
difference between $A$ and $C$ is to $A$ as the difference between $C$ and $B$
is to $B$. Permuting, the difference between $A$ and $C$ is to the difference
between $C$ and $B$ as $A$ is to $C$. Now putting these together, the difference
between $A$ and $C$ plus the difference between $C$ and $B$, that is, the
difference between $A$ and $B$ is to the difference between $C$ and $B$ as $A+C$
is to $C$. But $A+C$ is to $C$ as $2C$ is to $D$ and the difference between $A$
and $B$ is to the difference between $C$ and $B$ as $2C$ is to $D$. Since $A+C$
is to $C$ as $2C$ is to $D$, permuting gives $A+C$ is to $2C$ as $C$ is to $D$.
Dividing, the difference between $A$ and $C$ is to $2C$ as the difference
between $C$ and $D$ is to $D$. Again permuting, the difference between $A$ and
$C$ is to the difference between $C$ and $D$ as $2C$ is to $D$. But now the
difference between $A$ and $B$ has been demonstrated to be to the difference
between $C$ and $B$ as $2C$ is to $D$, and from this the difference between $A$
and $B$ is to the difference between $C$ and $B$ as the difference between $A$
and $C$ is to the difference between $C$ and $D$. However the difference between
$A$ and $B$ is greater than the difference between $C$ and $B$ and the
difference between $A$ and $C$ is greater than the difference between $C$ and
$D$. Permuting the previous ratio, the difference between $A$ and $B$ is to the
difference between $A$ and $C$ as the difference between $C$ and $B$ is to the
difference between $C$ and $D$. The difference between $A$ and $B$ is greater
than the difference between $A$ and $C$ and the difference between $C$ and $B$
is greater than the difference between $C$ and $D$, and the difference between
$A$ and $B$ is equal to the difference between $A$ and $C$ plus the difference
between $C$ and $B$. Therefore either of them is greater than the difference
between $C$ and $D$ and it is obvious that the difference between $A$ and $B$ is
greater than the double of the difference between $C$ and $D$, that is, the
difference between the triangle $ABP$ and the quadrilateral $ABFP$ is greater than twice
the difference between the quadrilateral $ABIP$ and the polygon $ABDLP$, Q.E.D.

\begin{samepage}
\begin{center}
\large\textsc{Scholium.}
\end{center}

The same method may by all means be used to demonstrate that the difference
between quadrilateral $ABIP$ and polygon $ABDLP$ is greater that twice the
difference between polygons $ABEIOP$ and $ABCGKNP$. From here by the same method
one is able to demonstrate this difference is always exceeded in our doubling to
infinity of the complex polygon. In fact the difference between the prior
inscribed and circumscribed polygons is always greater than twice the difference
between the subsequent inscribed and circumscribed polygons. Thus it
bears more than half of the difference of the prior polygons to that of the
difference of the subsequent. Therefore continuing the subdoubling of the
polygon, we discover the two complex polygons, where the difference is made less
than whatever magnitude, as we assumed in the preceding Scholium.
\end{samepage}

Let there be two magnitudes, $a$ and $b$, with $a$ less than $b$, and let there
be two inequalities $c$ is greater than $d$ and $c$ is greater than $e$. From
here he have that $c$ is to $d$ as $b - a$ is to $\frac{bd - ad}{c}$ to which
$a$ is then added to the magnitude so that $\frac{ca + bd - ad}{c}$, where one
immediately assigns the magnitude as $a$. And also $c$ is to $e$ as $b - a$ is
to $\frac{be - ae}{c}$, where the magnitude is subtracted from $b$ yielding
$\frac{bc - be + ae}{c}$ which is then assigned to $b$.

One may continue the convergent series from here in which the first terms are
$a, b$, the second $\frac{ca+bd-ad}{c}, \frac{bc-be+ae}{c}$, and it is obvious
that the term $\frac{ca+bd-ad}{c}$ is greater than the term $a$ since the term
$a$ is added to $\frac{bd-ad}{c}$ giving $\frac{ca+bd-ad}{c}$. It is also clear
that the term $\frac{ca+bd-ad}{c}$ is less than the term $b$ since the
difference between $a$ and $b$ is greater than the difference between $a$ and
$\frac{ca+bd-ad}{c}$. It is evident that the term $\frac{bc-be+ae}{c}$ is less
that the term $b$ since $\frac{be-ae}{c}$ is subtracted from $b$ giving
$\frac{bc-be+ae}{c}$. And it is further obvious that the term
$\frac{bc-be+ae}{c}$ is greater than $a$ since the difference between $a$ and
$b$ is greater than the difference between $\frac{bc-be+ae}{c}$ and $b$.
Therefore it is evident that the difference between the convergent terms $a$ and
$b$ is greater than the difference between the convergent terms
$\frac{ca+bd-ad}{c}$ and $\frac{bc-be+ae}{c}$.

However since the convergent terms $a$ and $b$ where given as indefinite, $a$
and $b$ can be selected to be in the location of whichever of the convergent
terms of the whole of the series. So by putting $a$ and $b$ for whichever terms
of the convergent series, it necessarily follows from the composition of the
series that $\frac{ca+bd-ad}{c}$ and $\frac{bc-be+ae}{c}$ are the immediately
following convergent terms. And again since the difference between the terms $a$
and $b$ is greater than the difference between the terms $\frac{ca+bd-ad}{c}$
and $\frac{bc-be+ae}{c}$, it is clear that the difference between the
prior convergent term is always greater than the difference between the
subsequent convergent term. Because this difference always diminishes
proportionally in the ratio $b - a$ is to $\frac{bc - be + ae - ca - bd +
ad}{c}$, one can see that the terms of this convergent series are progressively
less. Therefore imagining this series continuing to infinity, we are able to
imagine the final convergent terms being equal, where we call these equal terms
the limit of the series.

\begin{samepage}
\begin{center}
\large\textsc{Proposition VII. Problem.}

\vskip 1.0em
\emph{To find the limit of the aforementioned series.}
\end{center}
\end{samepage}

So that this set of problems may be satisfied, we want to first find the
magnitude that is composed by the same method from the convergent terms $a$ and
$b$ as from the convergent terms $\frac{ca+bd-ad}{c}$ and $\frac{bc-be+ae}{c}$,
which follows easily from the following method. The magnitude may be obtained by
multiplication by $a$ and addition by $b$ times a magnitude $m$, and the same may
be obtained by multiplication by $\frac{ca+bd-ad}{c}$ and addition by
$\frac{bc-be+ae}{c}$ times a magnitude $m$. Let the magnitude be $z$, then $za+bm$
is equal to $\frac{zca + zbd - zad + mbc - mbe + mae}{c}$ and the equation
reduces to $z = \frac{mae - mbe}{ad - bd}$. This magnitude whether multiplied by
$a$ and added to $mb$, or multiplied by $\frac{ca+bd-ad}{c}$ and added to
$\frac{mbc-mbe+mae}{c}$ produces the same magnitude in either case, namely
$\frac{maae - mbae + mbad - mbbd}{ad - bd}$. And so the aforementioned
magnitude is composed by the same method from the convergent terms $a$ and $b$
as from the convergent terms $\frac{ca+bd-ad}{c}$ and $\frac{bc-be+ae}{c}$, and
because $a$ and $b$ are indefinite magnitudes they can be any convergent terms
whatsoever of the series, where the convergent terms immediately following are
$\frac{ca+bd-ad}{c}$ and $\frac{bc-be+ae}{c}$. Thus the magnitude $\frac{maae -
mbae + mbad - mbbd}{cd - bd}$ is composed by the same method from any of the
convergent terms of the series what are composed of the convergent terms $a$ and
$b$. Therefore the aforementioned magnitude is composed by the same method from
its final convergent terms, which are equal. Let this final term be $x$, which
multiplied by $\frac{mae - mbe}{ad - bd}$ and by $m$ produces $xm$ and
$\frac{xmae - xmbe}{ad - bd}$. Summing the factors yields $\frac{xmae - xmbe
+ xmad - xmbd}{ad - bd}$ is equal to $\frac{maae - mbae + mbad - mbbd}{ad -
bd}$, and the equation reduces to $x$ is equal to the term $\frac{aae - bae +
bad - bbd}{ae - be + ad - bd}$, which we wanted to find.

In order to make this problem less obscure by an exercise, we illustrate in
numbers: Let $c = 7$, $d = 2$, $e = 3$, $a = 28$, and $b = 42$. Then the second
convergent terms are $32$ and $36$, the third are $33\frac{1}{7}$ and
$34\frac{2}{7}$, and the limit is $33\frac{3}{5}$.

Changing nothing, if $a$ is less than $b$ then $\frac{ca+bd-ad}{c}$ may be
greater than $\frac{bc-be+ae}{c}$, indeed greater can be subtracted analytically
by lesser, which will not bear showing in the example. Let $c = 7$, $d
= 5$, $e = 4$, $a = 28$, and $b = 42$. The second convergent terms will be $38$
and $34$, the third $35\frac{1}{7}$ and $36\frac{2}{7}$, and the limit
$35\frac{7}{9}$.

The solution to this problem may even be obtained by this same method is $a$ is
zero, or exactly nothing. For example, let $c = 8$, $d = 3$, $e = 4$, $a = 0$,
and $b = 24$. Then the second convergent terms will be $9$ and $12$, the third
$10\frac{1}{8}$ and $10\frac{1}{2}$, and the limit of the series
$10\frac{2}{7}$.

Indeed the limits of these series can be found in Gregorie of St. Vincent's book
on geometric progression, although his way of proceeding differs greatly from
the one presented here.

\newpage
\begin{samepage}
\begin{center}
\large\textsc{Proposition VIII. Problem.}

\vskip 1.0em
\emph{Let the two quantities $A$ and $B$ be given and $C:D$ be any given
ratio. We want to find another magnitude so that the ratio of it to $A$ is
the multiplicate of $B:A$ in the ratio $C:D$.}\footnote{If a ratio $x$ is the
``multiplicate'' of the ratio $y$ in the ratio $z$, then in modern notation $x =
y^{z}$. Likewise, if a number $x$ is the ``submultiplicate'' of the ratio $y$ in
the ratio $z$, then in modern notation $x = y^{\frac{1}{z}}$. See
\cite[p.286]{Hutton}.}
\end{center}
\end{samepage}

First, let the ratio $C:D$ be commensurable, and let $E$ be a common measure of
$C$ and $D$. For as often as $E$ is contained in $D$ let the ratio $F:A$ be the
submultiplicate of $B:A$ in such ratio\footnote{That is, $\frac{F}{A} =
\left(\frac{B}{A}\right)^{\frac{E}{D}}$}. Also as often as $E$ is contained in
$C$ let the ratio $G:A$ be the multiplicate of the ratio $F:A$ in such
ratio\footnote{That is, $\frac{G}{A} = \left(\frac{B}{A}\right)^{\frac{C}{E}}$}.
I claim that $G$ is the desired magnitude. The ratio $G:A$ is the multiplicate
of the ratio $F:A$ in the ratio $C:E$, and the ratio $F:A$ is the multiplicate
of the ratio $B:A$ in the ratio $E:D$. Therefore by equality, the ratio $G:A$ is
the multiplicate of the ratio $B:A$ in the ratio $C:D$, which is what we wanted
to show.

If the ratio $C:D$ is incommensurable, then I am convinced that in practice this
problem is geometrically impossible. However it can be accomplished by
approximation, assuming a commensurable ratio that approaches it.

Let there be a convergent series such that the first terms are $A$ and $B$, the
second $C$ and $D$, and the third $E$ and $F$. Let the second terms be made
by the first, where $B$ is greater than $A$, as the multiplicate of the ratio
$C:A$ in the ratio of $M:N$, with $M\geq N$, and the ratio of $B:A$ is the
multiplicate of the ratio $D:A$ in the ratio $M:O$, with $M\geq O$. Further, the
third terms are made from the second as the second are made from the first, and
so the series continues.

\begin{samepage}
\begin{center}
\large\textsc{Proposition IX. Problem.}

\vskip 1.0em
\emph{To find the limit of the aforementioned series.}
\end{center}
\end{samepage}

Set $G=0$, that is the exponent of the ratio of equality, or of the ratio $A:A$.
Also let $H$ satisfy the exponent of the ratio $B:A$. Let $M:N$ be as the
difference between $G$ and $H$, that is $H$ itself, or the exponent of the ratio
$B:A$, is to the excess of $I$ over $G$, that is $I$ itself, but
$M:N$ is the ratio by which $B:A$ is the multiplicate of the ratio $C:A$. Therefore
the excess of $I$ over $G$, that is $I$ itself, is the exponent of
the ratio $C:A$. Let $M:O$ be as the excess of $H$ over $G$, that is
$H$, is to the excess of $K$ over $G$, that is $K$, but $M:O$ is the
ratio by which $B:A$ is the multiplicate of the ratio $D:A$. Whenever $H$ is the
exponent of the ratio $B:A$, $K$ will be the exponent of the ratio $D:A$.
Therefore if $I$ is the exponent of the ratio $C:A$ and $K$ is the exponent of
the ratio $D:A$, then the excess of $K$ over $I$ will be the exponent
of the ratio $D:C$. From here let $M:N$ be as the excess of $K$ over
$I$, or the exponent of the ratio $D:C$, is to the excess of $R$ over
$I$, but $M:N$ is the ratio, from the composition of the series, by which $D:C$
is the multiplicate of $E:C$, and so the excess of $K$ over $I$ is the
exponent of the ratio $D:C$. Thus the excess of $R$ over $I$ is the
exponent of the ratio $E:C$ and $I$ is the exponent of the ratio $C:A$.
Therefore $R$ is the exponent of the ratio $E:A$. From here let $M:O$ as the
excess of $K$ over $I$ is to the excess of $S$ over $I$,
but $M:O$ is the ratio, from the composition of the series, by which $D:C$ is 
the multiplicate of $F:C$, where the excess of $K$ over $I$ is the
exponent of the ratio $D:C$. The excess of $S$ over $I$ will be the
exponent of the ratio $F:C$ and $I$ is the exponent of the ratio $C:A$. Thus $S$
is the exponent of the ratio $F:A$. Therefore when $R$ is the exponent of $E:A$
and $S$ is the exponent of the ratio $F:A$, the excess of $S$ over
$R$ will be the exponent of the ratio $F:E$. Continuing whichever series, it may
be demonstrated as before that $T$ be the exponent of $X:A$ and $V$ the exponent
of the ratio $Y:A$. Finally it will always be shown that the convergent terms of
the series of exponents are exponents of the ratios, and specifically of the
convergent terms of the proposed series by the first magnitude, $A$ of the
series, of whichever convergent terms of the series may be found in the same way
by the initial values. Thus by the term of the series of exponents through this
7 found. For example, let $L$, it will be the exponent of the ratio, be the
limit of the proposed series with the first term $A$. Therefore $Z:A$ may be
found that is the multiplicate of the given $B:A$ in the ratio $L:H$, and $Z$
will be the desired limit, which we wanted to find.

To illustrate this problem in numbers, let $M = 4$, $N = 2$, $O = 1$, $A = 6$,
and $B = 10$. Then the second convergent terms shall be $\sqrt{60}$ and
$\left(2160\right)^{\frac{1}{4}}$, the third convergent terms
$\left(7776000\right)^{\frac{1}{8}}$ and
$\left(100776960000000\right)^{\frac{1}{16}}$, and the limit of the series
$\left(360\right)^{\frac{1}{3}}$.

As another example, let $M = 6$, $N = 2$, $O = 3$, $A = 5$, and $B = 10$. Then
the second convergent terms of the series shall be
$\left(250\right)^{\frac{1}{3}}$ and $\sqrt{50}$, the third
$\left(488281250000000\right)^{\frac{1}{18}}$ and
$\left(7812500000\right)^{\frac{1}{12}}$, and the limit of the series
$\left(12500\right)^{\frac{1}{5}}$. Thus far all of the limits of the convergent
series can be made either by a single arithmetic proportion or a single
geometric proportion. Now I shall add to the method, and by the power of this
the limits of all convergent series may be found.

\newpage
\begin{samepage}
\begin{center}
\large\textsc{Proposition X. Problem.}

\vskip 1.0em
\emph{To find the limit of a given series from a given magnitude composed by 
the same method from two convergent terms of any convergent series in the same
way as from the subsequent convergent terms of the same series.}
\end{center}
\end{samepage}

Let the convergent series be of any two convergent terms $a$ and $b$ and the the
subsequent convergent terms $\sqrt{ab}$ and $\frac{aa}{\sqrt{ab}}$. The sum
of the convergent terms $a+b$ multiplied by the first convergent term $a$ gives
$aa+ab$. The sum of the subsequent convergent terms
$\sqrt{ab}+\frac{a^{2}}{\sqrt{ab}}$ multiplied by the first convergent term
$\sqrt{ab}$ likewise gives $aa+ab$. From this is discovered the limit of
the convergent series. It is clear that the magnitude $aa+ab$ is made by
the same method from the convergent terms $a$ and $b$ as from the subsequent
convergent terms $\sqrt{ab}$ and $\frac{aa}{\sqrt{ab}}$, and because the
magnitudes $a$ and $b$ were arbitrarily chosen terms of the convergent series,
it is evident that the sum of any proposed convergent terms of the series
multiplied by the first convergent term will give that same magnitude, which is
likewise the sum of the subsequent convergent terms multiplied by the first
convergent term. Since two convergent terms are always followed by two
convergent terms, it is clear that the sum of any two of the convergent
terms multiplied by the first convergent term will be $aa+ab$. And so the
final convergent terms are equal. Therefore let the final term of this series be
the limit $z$, which is added to itself and the sum multiplied by itself to give
$2zz$, which equals the magnitude $aa+ab$, and solving this equation
for $z$ yields the limit of the series $\sqrt{\frac{aa+b^{2}}{2}}$, which we
wanted to find.

And therefore in order to find the limit of any convergent series it is
necessary only to discover a magnitude composed by the same method from the
first convergent terms as is likewise composed from the second convergent
terms.

\begin{samepage}
\begin{center}
\large\textsc{Conclusions.}
\end{center}

Since it is not important to the problem whether the convergent terms $a$ and
$b$ are the first, second, third, etc., it is clear that all of the convergent
terms of the series are composed by the same method from the first convergent
terms as by the second, third, fourth, etc. convergent terms.
\end{samepage}

\begin{samepage}
\begin{center}
\large\textsc{Proposition XI. Theorem.}

\vskip 1.0em
\emph{The sector of the circle, ellipse, or hyperbola $ABIP$ is not composed
analytically by the triangle $ABP$ and the quadrilateral $ABFP$.}
\end{center}
\end{samepage}

Let the triangle $ABP = a$ and the quadrilateral $ABFP = b$. It is clear from the
preceding propositions that the quadrilateral $ABIP = \sqrt{ab}$ and the polygon $ABDLP
= \frac{2ab}{a+\sqrt{ab}}$. The sector $ABIP$ is the limit of this convergent
series. So that the signs of radicals and fractions may be removed from the
terms of the series, for the first convergent terms of the series $a$ and $b$,
that is, for the triangle $ABP$ and the quadrilateral $ABFP$, put $a^{3}+a^{2}b$
and $a^{2}b+b^{3}$. Then the second convergent terms of the series, that is, the
quadrilateral $ABIP$ and the polygon $ABDLP$, will be $ba^{2}+b^{2}a$ and $2b^{2}a$. I
claim that the limit of the convergent series (where the first convergent terms
of the series are $a^{3}+a^{2}b$ and $a^{2}b+b^{3}$ and the second are
$ba^{2}+b^{2}a$ and $2b^{2}a$) is not composed analytically of the terms
$a^{3}+a^{2}b$ and $a^{2}b+b^{3}$. Indeed, if the aforementioned limit is
composed analytically of the terms $a^{3}+a^{2}b$ and $a^{2}b+b^{3}$, then the
limit would itself be analytic and would be composed by the same method from
the convergent terms $ba^{2}+b^{2}a$ and $2b^{2}a$. Therefore the limit would
be composed analytically by the same method from $a^{3}+a^{2}b$ and
$a^{2}b+b^{3}$ as it is composed from $ba^{2}+b^{2}a$ and $2b^{2}a$, but no
magnitude may be composed analytically by the same method from $a^{3}+a^{2}b$
and $a^{2}b+b^{3}$ as it is composed from $ba^{2}+b^{2}a$ and $2b^{2}a$, which
I thus demonstrate. If a magnitude may be composed analytically by the same
method from $a^{3}+a^{2}b$ and $a^{2}b+b^{3}$ as it is composed from
$ba^{2}+b^{2}a$ and $2b^{2}a$, then the same magnitude would be made by adding,
subtracting, multiplying, dividing, and extracting roots from the terms
$a^{3}+a^{2}b$ and $a^{2}b+b^{3}$ as if by the same method the terms
$ba^{2}+b^{2}a$ and $2b^{2}a$ were added, subtracted, multiplied, divided, and
roots extracted. However the latter is not possible to do, so neither can be
the former. Thus I prove less***, if the same magnitude is made by addition,
subtraction, multiplication, division, and root extraction of the terms
$a^{3}+a^{2}b$ and $a^{2}b+b^{3}$, which themselves are made by addition,
subtraction, multiplication, division, and root extraction of the terms
$ba^{2}+b^{2}a$ and $2b^{2}a$, then by adding, or subtracting, or multiplying,
or dividing equal magnitudes by or to the terms $a^{3}+a^{2}b$ and
$a^{2}b+b^{3}$, or by root extraction, these analytic operations turning into
others, by reiterating any of them or doing none of them, the two terms can be
made into the final product, one from the term $a^{2}b+b^{3}$ and the other from
the term $2b^{2}a$, so that the final product from the term $a^{3}+a^{2}b$ with
the final product from $a^{2}b+b^{3}$ is the same as the final product by the
term $ba^{2}+b^{2}a$ with the final product from the term $2b^{2}a$ in the same
way added, subtracted, multiplied, divided, and roots extracted.

\noindent\ldots Yikes! \ldots

\noindent And therefore these two magnitudes cannot be equal, where the other may be
obtained by $a$ itself in one as in the other. And so it is evident that a
sector of a circle, ellipse, or hyperbola $ABIP$ cannot be composed analytically
from the triangle $ABP$ and the quadrilateral $ABFP$, QED.

However, so that the purpose may be made evident, I subject it to another more
brief and easier proof stemming from another means of attack. A magnitude cannot
be composed analytically from the terms $a^{3}+a^{2}b$ and $a^{2}b+b^{3}$ by the
same method as the same magnitude is composed from the terms $ba^{2}+b^{2}a$ and
$2b^{2}a$ because by adding, subtracting, multiplying, and dividing
$a^{3}+a^{2}b$ and $a^{2}b+b^{3}$, and by extracting roots, 


\begin{samepage}
\begin{center}
\large\textsc{Scholium.}
\end{center}

It will be seen as most obscure\ldots
\end{samepage}

\begin{center}
\large\textsc{Proposition XII. Theorem.}
\end{center}

\begin{figure}[h]
\includegraphics[scale=0.95]{vera_i}
\end{figure}

\vskip 1.0em
Let the quadrilateral $ABIP$ be $A$, the polygon $ABEIOP$ be $C$, the polygon $ABCGKNP$
be $D$, and the polygon $ABDLP$ be $B$. I claim that $D$ is the harmonic mean of
$C$ and $B$. Combining proposition 4 and $A:C::C:B$ gives $A+C:C::C+B:B$. But
then from proposition 5 $A+C:C::2C:D$ and also $C+B:B::2C:D$. Permuting gives
$B+C:2C::B:D$ and by dividing, the difference between $B$ and $C$ is to $2C$ as
the difference between $B$ and $D$ is to $D$. Again by permuting, the difference
between $B$ and $C$ is to the difference between $B$ and $D$ as $2C$ is to $D$,
that is, as $C+B$ is to $B$. And by dividing, the difference between $D$ and $C$
is to the difference between $B$ and $D$ as $C$ is to $B$. Therefore $D$ is the
harmonic mean between $C$ and $B$, Q.E.D.

This proposition is valid by the same method for every complex polygon, so that
it follows from the Scholium of Proposition 5.

\begin{center}
\large\textsc{Proposition XIII. Theorem.}
\end{center}

\vskip 1.0em
Let C be the arithmetic mean, D the geometric mean, and E the harmonic
mean of the two magnitudes $A$ and $B$. I claim that $C$, $D$, and $E$ are
continuously proportional\footnote{That is, $C:D::D:E$.}. Because $A$, $E$, and
$B$ are in harmonic ratio, the difference between $A$ and $E$ shall be to the difference between $E$ and $B$ as
$A$ is to $B$. By combining, the difference between $A$ and $B$ shall be to the
difference between $E$ and $B$ as $A+B$ is to $B$. From here, by permuting and
combining, $2A:A+B::E:B$, but $2A$ is twice $A$ and $A+B$ is twice $C$, so that
$A:C::E:B$. Thus $CE=AB$ and $AB=DD$, and so $CE=DD$. Therefore $C:D::D:E$,
Q.E.D.

\begin{figure}[h]
\includegraphics[scale=0.95]{vera_inscr}
\caption*{Series of inscribed polygons.}
\end{figure}

\begin{center}
\large\textsc{Proposition XIV. Theorem.}
\end{center}

\vskip 1.0em
Let $A$ and $B$ be two complex polygons with $A$ inscribed in the sector of a
circle or ellipse and $B$ circumscribed. A convergent series of these complex
polygons may be continued according to our method of drawing the subdouble, so
that the polygons inscribed in the circle are $A$, $C$, $E$, etc. and those
circumscribed are $B$, $D$, $F$, etc. I claim that $A+E$ is less than $2C$.
This is clear from the previous propositions by the following analogies.
First, $A$, $C$, and $B$ are continuously proportional, and second $C$, $D$, and
$B$ are in harmonic proportion. Therefore the excess of $C$ over $A$,
that is $C-A$, is to the excess of $D$ over $C$, or $D-C$, in a ratio
composed from the proportion $A:C$ and from the proportion $A+C:A$, that is in
the ratio $A+C:C$. And $A+C$ is greater than $C$, so that the excess of
$C$ over $A$ is greater than the excess of $D$ over $C$.
However $D$ is greater than $E$ and so the excess of $C$ over $A$ is
much greater than the excess of $E$ over $C$. Therefore $A+E$ is less
than $2C$, Q.E.D.

\begin{figure}[h]
\includegraphics[scale=0.95]{vera_circumscr}
\caption*{Series of circumscribed polygons.}
\end{figure}

\begin{center}
\large\textsc{Proposition XV. Theorem.}
\end{center}

\vskip 1.0em
By the same assumptions, I claim that the excess of $C$ over $A$ is
less than the quadruple of the excess of $E$ over $C$. This is
clear from the previous propositions by analogy with the following three
analogies. First, $A$, $C$, and $B$ are continuously proportional, second $C$, $D$, and
$B$ are in harmonic proportion, and third $C$, $E$, and $D$ are continuously
proportional. And so the excess of $C$ over $A$, that is $C-A$, is to
the excess of $E$ over $C$, or $E-C$, as $AC+EC+AE+CC$ is to $CC$,
and $B$ is greater than $E$. So $AB$, or $CC$, is greater than $AE$, and thus
$AE+CC$ is less than $2CC$. And so $AC+EC$ is to $2CC$ as $A+E$ is to $2C$, but
$A+E$ is less than $2C$ so that $AC+EC$ is less than $2CC$. Thus $AC+EC+AE+CC$
is less than $4CC$. Therefore $C-A$ is less than the quadruple of $E-C$, Q.E.D.

\begin{center}
\large\textsc{Proposition XVI. Theorem.}
\end{center}

\vskip 1.0em
Let $A$ and $B$ be two complex polygons with $A$ circumscribed about the sector
of a hyperbola and $B$ inscribed. A convergent series of these complex polygons
may be continued according to our method of drawing the subdouble, so that the
polygons circumscribed about the hyperbola are $A$, $C$, $E$, etc. and those
inscribed are $B$, $D$, $F$, etc. I claim that $A+E$ is greater than $2C$. This
is clear from the previous propositions by the following analogies. First, $A$,
$C$, and $B$ are continuously proportional, and second $C$, $D$, and $B$ are in
harmonic proportion. Therefore the excess of $A$ over $C$, that is
$A-C$, is to the excess of $C$ over $D$, or $C-D$, in a ratio
composed from the proportion $A:C$ and from the proportion $A+C:A$, that is in
the ratio $A+C:C$. And $A+C$ is greater than $C$, so that the excess of
$A$ over $C$ is greater than the excess of $C$ over $D$. However
$E$ is greater than $D$ and so the excess of $A$ over $C$ is much
greater than the excess of $C$ over $E$. Therefore $A+E$ is greater
than $2C$, Q.E.D.

\begin{center}
\large\textsc{Proposition XVII. Theorem.}
\end{center}

\vskip 1.0em
By the same assumptions, I claim that the excess of $A$ over $C$ is
greater than the quadruple of the excess of $C$ over $E$. This is
clear from the previous propositions by analogy with the following three
analogies. First, $A$, $C$, and $B$ are continuously proportional, second $C$, $D$, and
$B$ are in harmonic proportion, and third $C$, $E$, and $D$ are continuously
proportional. And so the excess of $A$ over $C$, that is $A-C$, is to
the excess of $C$ over $E$, or $C-E$, in a ratio composed of the
proportions $A:C$, $A+C:A$, and $E+C:C$. And so $A-C$ is to $C-E$ as
$AC+EC+AE+CC$ is to $CC$, and $B$ is less than $E$. So $AB$, or $CC$, is less
than $AE$, and thus $AE+CC$ is greater than $2CC$. And so $AC+EC$ is to $2CC$ as
$A+E$ is to $2C$, but $A+E$ is greater than $2C$ so that $AC+EC$ is greater than
$2CC$. Thus $AC+EC+AE+CC$ is greater than $4CC$. Therefore $C-A$ is greater than
the quadruple of $E-C$, Q.E.D.

\begin{center}
\large\textsc{Proposition XVIII. Theorem.}
\end{center}

\vskip 1.0em
Let $A$ and $B$ be two magnitudes such that $A$ is less than $B$. Let $C$ be
their geometric mean and $D$ their arithmetic mean. I claim that $D$ is greater
than $C$. Since $B$, $C$, and $A$ are continuously proportional, by dividing,
permuting, and combining, it shall be that the excess of $B$ over $A$
is to the excess of $C$ over $A$ as $A+C$ is to $A$. And so $A+C$ is
greater than twice $A$. Thus the excess of $B$ over $A$ is greater
than twice the excess of $D$ over $A$, so that the excess of
$D$ over $A$ is greater than the excess of $C$ over $A$.
Therefore $D$ is greater than $C$, Q.E.D.

\begin{center}
\large\textsc{Proposition XIX. Theorem.}
\end{center}

\vskip 1.0em
By the same assumptions, let $E$ be the harmonic mean of $A$ and $B$. I claim
that $C$ is greater than $E$. From proposition 13, $D$ is to $C$ as $C$ is to
$E$, but $D$ is greater than $C$. Therefore $C$ is greater than $E$, Q.E.D.

\begin{center}
\large\textsc{Conclusion.}
\end{center}

\vskip 1.0em
From the two preceding propositions it is obvious that $D$ is greater than $E$,
that is, that the arithmetic mean of two magnitudes is greater than the harmonic
mean of the same.

\begin{center}
\large\textsc{Proposition XX. Theorem.}
\end{center}

\vskip 1.0em
Let $A$ and $B$ be two complex polygons with $A$ inscribed in the sector of a
circle or ellipse and $B$ circumscribed. A convergent series of these complex
polygons may be continued according to our method of drawing the subdouble, so
that the polygons inscribed in the circle are $A$, $C$, $E$, $K$, etc. and those
circumscribed are $B$, $D$, $F$, $L$, etc. Also let $Z$ be the limit of the
convergent series, that is, the sector of the circle or ellipse. I claim that
$Z$ is greater than $C$ plus one third of the excess of $C$ over $A$.
Let the excess of $G$ over $C$ be a fourth part of the excess of
$C$ over $A$ and the excess of $H$ over $G$ be a fourth part of
the excess of $G$ over $C$. This series may be continued
infinitely, so let $X$ be the limit of this process. The excess of $C$
over $A$ is less than the quadruple of the excess of $E$ over
$C$, and so the excess of $E$ over $C$ is greater than the excess
of $G$ over $C$, and therefore $E$ is greater than $G$. Now the
excess of $E$ over $C$ is less than the quadruple of the excess of $K$
over $E$, and so the excess of $G$ over $C$ is much less than the excess of $K$
over $E$, and therefore the excess of $K$ over $E$ is greater than the excess of $H$ over $G$.
Since $E$ is greater than $G$, it is obvious $K$ is greater than $H$. It is
demonstrated by the same method in every series $A$, $C$, $E$ and $A$, $C$, $G$,
by continuation to however many terms. Each term of the series $A$, $C$, $E$
is greater than the corresponding term of the series $A$, $C$, $G$. And so the
limit of the series $A$, $C$, $E$, that is $Z$, will be greater than the limit
of the series $A$, $C$, $G$, that is $X$. And from Archimedes the quadrature of
the parabola fixed as $X$ is equal to $C$ plus one third of the excess of $C$
over $A$, and therefore $Z$ is greater than it, Q.E.D.

\begin{center}
\large\textsc{Proposition XXI. Theorem.}
\end{center}

\vskip 1.0em
By the same assumptions as above, I claim that $Z$, which is a sector of a
circle or ellipse, is less than the greater of the two continuously
proportional arithmetic means of $A$ and $B$. Let $G$ be the arithmetic mean
of $A$ and $B$ and $H$ be the arithmetic mean between $G$ and $B$. Likewise let $M$
be the arithmetic mean of $G$ and $H$ and $N$ be the arithmetic mean of $M$ and
$H$. This convergent series, with terms $AB$, $GH$, $MN$,
$OP$, may be continued infinitely, so that its limit is $X$. It is clear
from the preceding propositions that $G$ is greater than $C$, and $H$, the
arithmetic mean of $G$ and $B$, is greater than the harmonic mean of $G$ and
$B$. However the harmonic mean of $G$ and $B$ is greater than $D$, the harmonic
mean of $C$ and $B$, since $G$ is greater than $C$. And so the arithmetic mean
of $G$ and $B$, that is $H$, is greater than $D$, the harmonic mean of $C$ and
$B$. By the same method $M$, the arithmetic mean of $G$ and $H$ is greater than
the geometric mean between $G$ and $H$. And since $G$ is greater than $C$, and
$H$ is greater than $D$, the geometric mean of $G$ and $H$ is greater than $E$,
the geometric mean of $C$ and $D$. Thus $M$ is greater than $E$. Now $N$, the
arithmetic mean of $M$ and $H$, is greater than the harmonic mean of the same,
and since $H$ is greater than $D$ and $M$ is greater than $E$, the harmonic mean
of $M$ and $H$ is greater than $F$, the harmonic mean of $E$ and $D$. And so $N$
is greater than $F$. Continuing the series by the same method to infinity, one
may always show that the terms of the series $AB$, $CD$ are less
than the corresponding terms of the series $AB$, $GH$. Therefore
the limit, $Z$, of the series $AB$, $CD$ will be less than the
limit, $X$, of the series $AB$, $GH$. Also, from Proposition 7, the limit, $X$,
of the series $AB$, $GH$ is equal to the greater of the two
continuously proportional arithmetic means of $A$ and $B$, and so $Z$ is the
less than the same, Q.E.D.

\begin{center}
\large\textsc{Proposition XXII. Theorem.}
\end{center}

\vskip 1.0em
By the same assumptions as above, I claim that I claim that $Z$, which is a
sector of a circle or ellipse, is less than the greater of the two continuously
proportional geometric means of $A$ and $B$. Let $G$ be the geometric mean
of $A$ and $B$ and $H$ be the geometric mean between $G$ and $B$. Likewise let
$M$ be the geometric mean of $G$ and $H$ and $N$ be the geometric mean of $M$
and $H$. This convergent series, with terms $AB$, $GH$, $MN$,
$OP$, may be continued infinitely, so that its limit is $X$. It is clear
from the preceding propositions that $C$ and $G$ are equals, and $H$ is greater
than $D$. By this reasoning, $M$, the geometric mean of $G$ and $H$, is greater
than $E$, the geometric mean of $C$ and $D$. Now $N$, the geometric mean of $M$
and $H$, is greater than the harmonic mean of the same, and since $M$ is greater
than $E$ and $H$ is greater than $D$, the harmonic mean of $M$ and $H$ is
greater than $F$, the harmonic mean of $E$ and $D$. And so $N$ is greater than
$F$. Continuing the series by the same method to infinity, one may always show
that the terms of the series $AB$, $CD$ are less than the corresponding terms of
the series $AB$, $GH$. Therefore the limit, $Z$, of the series $AB$, $CD$ will
be less than the limit, $X$, of the series $AB$, $GH$. Also, from Proposition 9,
the limit, $X$, of the series $AB$, $GH$ is equal to the greater of the two
continuously proportional geometric means of $A$ and $B$, and so $Z$ is the
less than the same, Q.E.D.

\begin{center}
\large\textsc{Scholium.}
\end{center}

\vskip 1.0em
It is not much work for me to show that the greater of the two continuously
proportional arithmetic means of two unequal magnitudes is greater than the
greater of the two continuously proportional geometric means of the same
magnitudes. Therefore it is a more exact approximation of the previous
proposition, if it be carried out. However I use the preceding proposition for
its ease.

\begin{center}
\large\textsc{Proposition XXIII. Theorem.}
\end{center}

\vskip 1.0em
Let $A$ and $B$ be two complex polygons with $A$ circumscribed in the sector of
a hyperbola and $B$ inscribed. A convergent series of these complex
polygons may be continued according to our method of drawing the subdouble, so
that the polygons circumscribed in the circle are $A$, $C$, $E$, $K$, etc. and those
inscribed are $B$, $D$, $F$, $L$, etc. Also let $Z$ be the limit of the
convergent series, that is, the sector of the hyperbola. I claim that $Z$ is
greater than $C$ subtracted from one third of the excess of $A$ over
$C$. Let the excess of $C$ over $G$ be a fourth part of the excess of $A$ over
$C$ and the excess of $G$ over $H$ be a fourth part of the excess of $C$ over
$G$. This series may be continued infinitely, so let $X$ be the limit of this
process. The excess of $A$ over $C$ is greater than the quadruple of the excess
of $C$ over $E$, and so the excess of $C$ over $E$ is less than the excess of
$C$ over $G$, and therefore $E$ is greater than $G$. Now the excess of $C$ over
$E$ is greater than the quadruple of the excess of $E$ over $K$, and so the
excess of $C$ over $G$ is much greater than the excess of $E$ over $K$, and
therefore the excess of $G$ over $H$ is greater than the excess of $E$ over $K$.
Since $E$ is greater than $G$, it is obvious $K$ is greater than $H$. It is
demonstrated by the same method in every series $A$, $C$, $E$, $K$ and $A$, $C$,
$G$, $H$ by continuation to however many terms. Each term of the series $A$,
$C$, $E$ is greater than the corresponding term of the series $A$, $C$, $G$. And so the
limit of the series $A$, $C$, $E$, that is $Z$, will be greater than the limit
of the series $A$, $C$, $G$, that is $X$. And from Archimedes the quadrature of
the parabola fixed as $X$ is equal to $C$ plus one third of the excess of $C$
over $A$, and therefore $Z$ is greater than it, Q.E.D.

\begin{center}
\large\textsc{Proposition XXIV. Theorem.}
\end{center}

\vskip 1.0em
By the same assumptions as above, I claim that $Z$, which is a sector of a
hyperbola, is less than the lesser of the two continuously proportional
arithmetic means of $A$ and $B$. Let $G$ be the arithmetic mean of $A$ and $B$
and $H$ be the arithmetic mean between $G$ and $B$. Likewise let $M$ be the
arithmetic mean of $G$ and $H$ and $N$ be the arithmetic mean of $M$ and $H$.
This convergent series, with terms $AB$, $GH$, $MN$, $OP$, may be continued
infinitely, so that its limit is $X$. It is clear from the preceding
propositions that $G$ is greater than $C$, and $H$, the arithmetic mean of $G$
and $B$, is greater than the harmonic mean of $G$ and $B$. However the harmonic
mean of $G$ and $B$ is greater than $D$, the harmonic mean of $C$ and $B$, since
$G$ is greater than $C$. And so the arithmetic mean of $G$ and $B$, that is $H$,
is greater than $D$, the harmonic mean of $C$ and $B$. By the same method $M$,
the arithmetic mean of $G$ and $H$ is greater than the geometric mean between
$G$ and $H$. And since $G$ is greater than $C$, and $H$ is greater than $D$, the
geometric mean of $G$ and $H$ is greater than $E$, the geometric mean of $C$ and
$D$. Thus $M$ is greater than $E$. Now $N$, the arithmetic mean of $M$ and $H$,
is greater than the harmonic mean of the same, and since $H$ is greater than $D$
and $M$ is greater than $E$, the harmonic mean of $M$ and $H$ is greater than
$F$, the harmonic mean of $E$ and $D$. And so $N$ is greater than $F$.
Continuing the series by the same method to infinity, one may always show that
the terms of the series $AB$, $CD$ are less than the corresponding terms of the
series $AB$, $GH$. Therefore the limit, $Z$, of the series $AB$, $CD$ will be
less than the limit, $X$, of the series $AB$, $GH$. Also, from Proposition 7,
the limit, $X$, of the series $AB$, $GH$ is equal to the lesser of the two
continuously proportional arithmetic means of $A$ and $B$, and so $Z$ is the
less than the same, Q.E.D.

\begin{center}
\large\textsc{Proposition XXV. Theorem.}
\end{center}

\vskip 1.0em
By the same assumptions as above, I claim that I claim that $Z$, which is a
sector of a hyperbola, is less than the lesser of the two continuously
proportional geometric means of $A$ and $B$. Let $G$ be the geometric mean
of $A$ and $B$ and $H$ be the geometric mean between $G$ and $B$. Likewise let
$M$ be the geometric mean of $G$ and $H$ and $N$ be the geometric mean of $M$
and $H$. This convergent series, with terms $AB$, $GH$, $MN$,
$OP$, may be continued infinitely, so that its limit is $X$. It is clear
from the preceding propositions that $C$ and $G$ are equals, and $H$ is greater
than $D$. By this reasoning, $M$, the geometric mean of $G$ and $H$, is greater
than $E$, the geometric mean of $C$ and $D$. Now $N$, the geometric mean of $M$
and $H$, is greater than the harmonic mean of the same, and since $M$ is greater
than $E$ and $H$ is greater than $D$, the harmonic mean of $M$ and $H$ is
greater than $F$, the harmonic mean of $E$ and $D$. And so $N$ is greater than
$F$. Continuing the series by the same method to infinity, one may always show
that the terms of the series $AB$, $CD$ are less than the corresponding terms of
the series $AB$, $GH$. Therefore the limit, $Z$, of the series $AB$, $CD$ will
be less than the limit, $X$, of the series $AB$, $GH$. Also, from Proposition 9,
the limit, $X$, of the series $AB$, $GH$ is equal to the lesser of the two
continuously proportional geometric means of $A$ and $B$, and so $Z$ is the
less than the same, Q.E.D.

It is clear from this claim that this approximation is that one demonstratred
in the preceding proposition, if this one might be a little more laborious.
One will not ignore, however, that the two series can have equal limits, and
such that however many terms of one series would be always be greater than the
corresponding terms of the other series. But in such series that are carried out
a long way, the difference is less than it of the same number by means of the
number of terms. But the counter of our series that are carried out a long
way differs more greatly by the number of terms, as one can very easily show.

I observe by experiment that the difference between the second of two
proportional arithmetic means and the second of two proportional geometric means
is always much greater than the difference between the second of two
proportional geometric means and the sector of the circle, ellipse, or
hyperbola. That noted I consider appropriate, indeed this sector is
obtained differing scarcely beyond one from the second of the proportional
arithmetic means, when the arithmetic mean does not exceed the the geometric
mean beyond one, which exceedingly noted, for from this it is clear that the
approximation is boldly employed, when the series is continued to that the
midpoint of first of the noted is the same in either convergent term, which
experience likewise have evinced. In fact the sector never in this case differs
by unity from the second of the two continuously proportional arithmetic means.

Similarly another approximation is altogether most brief and most astonishing,
although it may not turn out to strengthen that geometric demonstration to me.
Namely if the first third of the noted in whichever term be converging to the
same, the sector of the circle, ellipse, or hyperbola always differs within
unity from the greatest quarter by the arithmetic continous proportion the other
of the terms of our approximation.

\begin{center}
\large\textsc{Proposition XXVI. Theorem.}
\end{center}

\vskip 1.0em
Let $CFN$ be any hyperbola with center $A$ and asymptotes $AB$ and $AO$. Also,
let $AFGL$ be its sector, with circumscribed triangle $AFL$. Let lines $FD$ and
$LM$ be drawn parallel to the asymptote $AB$ and complete parallelograms $FDMK$
and $PLMD$. I claim that the triangle $AFL$ is the arithmetic mean of
parallelograms $FDMK$ and $PLMD$. Gregorie of St. Vincent shows in his Libra de
Hyperbola that the triangle $AFL$ is equal to the quadrilateral $DFLM$, but it
is obvious that quadrilateral $DFLM$ is the arithmetic mean of parallelograms
$FDMK$ and $PLMD$, Q.E.D.

\begin{figure}[h!]
\includegraphics[scale=0.95]{vera_hyp_i}
\end{figure}

\begin{center}
\large\textsc{Proposition XXVII. Theorem.}
\end{center}

\vskip 1.0em
By the same assumptions, let line $AI$ be drawn bisecting $FL$ at $I$ and
intersecting the hyperbola at the point $G$. Also let $AFGL$ be a circumscribed
quadrilateral of the sector. I claim that this is the geometric mean of
parallelograms $FDMK$ and $PLMD$. From the proof of Gregorie of St. Vincent it
is evident that quadrilateral $AFGL$ is equal to $DFGLM$. Because $AGI$ bisects
the line $FL$ at $I$, from the Libra de Hyperbola of Gregorie of St. Vincent it
is clear that the lines $LM$, $GH$, and $FD$ are continually proportional in the
same ratio with the three continuous proportionals $AD$, $AH$, and $AM$. Let
the line $RGS$ be drawn through the point $G$ parallel to the asymptote $AO$,
meeting the lines $FD$ and $MK$ at the points $R$ and $S$. Because the lines
$FD$, $GH$, and $LM$ are continuously propotional, by dividing and permuting we
obtain $FR$ is to $SL$ as $GH$ is to $LM$. Likewise, since the lines $MA$, $HA$,
and $DA$ are continuously proportional, by dividing and permuting we
obtain $MH$ is to $HD$, that is $SG$ is to $GR$, as $HA$ is to $DA$, or $GH$ is
to $LM$. Thus $FR$ is to $SL$ as $SG$ it to $GR$, and when the angles $FRG$ and
$GLS$ are equal, on account of parallels $FR$ and $SL$ being equal, the
triangles $FRG$ and $GLS$ shall be equal. Therefore parallelogram $RDMS$ is
equal to polygon $DFGLM$, or quadrilateral $AFGL$. However, parallelogram $RDMS$
is the geometric mean of parallelograms $PDML$ and $FDMK$ since in having the
same height and the bases $LM$, $SM$ and $KM$ are continuously proportional. And
so the quadrilateral $AFGL$ is the geometric mean of parallelograms $PDML$ and
$FDMK$, Q.E.D.

\begin{center}
\large\textsc{Proposition XXVIII. Theorem.}
\end{center}

\vskip 1.0em
By the same assumptions, let the lines $FE$ and $LE$ be drawn tangent to the
hyperbola at points $F$ and $L$ in order to complete the quadrilateral $AFEL$. I
claim this is the harmonic mean of parallelograms $PDML$ and $FDMK$. Triangle
$AFL$, quadrilateral $AFGL$, and the harmonic mean of parallelograms $PDML$ and
$FDMK$ are continuously proportional since triangle $AFL$ is the arithmetic mean
and quadrilateral $AFGL$ the geometric mean of these parallelograms, as is clear
from Proposition 13. However triangle $AFL$, quadrilateral $AFGL$ and
quadrilateral $AFEL$ are continuously proportional by Proposition 11. Therefore
quadrilateral $AFEL$ is the harmonic mean of parallelograms $PDML$ and $FDMK$,
Q.E.D.

\begin{samepage}
\begin{center}
\large\textsc{Proposition XXIX. Problem.}

\vskip 1.0em
\emph{To find a square equal to a given circle.}
\end{center}
\end{samepage}

Let the square circumscribed by the circle be $4\cdot 10^{15}$, then the
inscribed square is $2\cdot 10^{15}$, between which \seqsplit{2828427124746190}
is the octagonal geometric mean. Now let the harmonic mean be between the
octagon inside the circle and the square about it, which by trivial labor is
found by dividing the double of the area of the octagonal inside the circle, or
the double of the rectangle of the areas inside and about the circle, by the sum
of the square and the octagon within. Then I find \seqsplit{3313708498984760} to
be the harmonic mean, the circumscribed octagon. Continuing this converging
series of the complex polygons where the midpoint of the first term is the same
in whichever convergent term, it is easy to do so up to the polygon of 16384
sides. Indeed the inscribed is \seqsplit{3141592576586860} and the circumscribed
\seqsplit{3141592692091258}. This is not considered the final term, since in
division and root extraction we always stray in some small part from the true
value, which the last imperfect term renders closely. From this is employed the
approximation from the proofs of Propositions 20 and 21, and the terms found
within will determine the true measure of the circle, positing the square of the
diamter as $4\cdot 10^{15}$, the smaller circle \seqsplit{3141592653589789}, and
the larger \seqsplit{3141592653589792}. Therefore the measure of the circle may
no longer hide, I deliver this series of polygons, which we wanted to find.

\vskip 1.0em
\begin{tabular}{ r p{129pt} p{129pt} }
      & Inside the circle & About the circle \\
4     & 2000000000000000 & 4000000000000000 \\
8     & 2828427124746190 & 3313708498984760 \\
16    & 3061467458920718 & 3182597878074527 \\
32    & 3121445152258051 & 3151724907429255 \\
64    & 3136548490545938 & 3144118385245904 \\
128   & 3140331156954752 & 3142223629942456 \\
256   & 3141277250932772 & 3141750369168965 \\
512   & 3141513801144299 & 3141632080703181 \\
1024  & 3141572940367090 & 3141602510256808 \\
2048  & 3141587725277158 & 3141595117749588 \\
4096  & 3141591421543029 & 3141593269613390 \\
8192  & 3141592345578073 & 3141592807595664 \\
16384 & 3141592576586860 & 3141592692091258
\end{tabular}

\vskip 1.0em
The circle consists of the following terms
\[3141592653589789 \qquad\qquad 3141592653589792\]
and by the same method altogether is obtained the equivalent polygon to
whichever circular or elliptic sector inscribed by a known triangle and
circumscribed by a quadrilateral.

\begin{samepage}
\begin{center}
\large\textsc{Proposition XXX. Problem.}

\vskip 1.0em
\emph{To find an arc from a given sine.}
\end{center}
\end{samepage}

Let $AE$ be the arc of the circle described about the center $B$. The radius of
this arc is of course $AB$ and the sine is $AD$. We want to find which
proportion of the arc itself has to the entire circumference of the circle. Let
$AE$ be the chord of the arc and its half-tangents be $AC$ and $CE$. From the
square of the radius $AB$ is produced the square of the sine $AD$, and square of
the cosine $BD$ remains, and thus $BD$ is given. Therefore the area of the
triangle $ABD$ is given by the rectangle and likewise the area of the triangle
$ABE$ is given, namely the rectangle of the given sine $AD$ and half of the
radius $BE$. From this it is seen that as the sum of the triangles $ABD$ and
$ABE$ is to the triangle $ABE$ as the double of the triangle $ABE$ is to the
quadrilateral $ABEC$, which is given by Proposition 5. From the given inscribed
triangle $ABE$ and the circumscribed quadrilateral $ABEC$ the sector $ABE$
itself may be found by the preceding Proposition, which to the given entire
circle has the desired proportion of the arc $AE$ to the total circumference,
which we wanted to find.

\begin{samepage}
\begin{center}
\large\textsc{Proposition XXXI. Problem.}

\vskip 1.0em
\emph{To find a sine from a given arc.}
\end{center}
\end{samepage}

\begin{figure}[h!]
\includegraphics[scale=0.95]{vera_ii}
\end{figure}

From the given arc it is clear how to give the area of the sector. Therefore
given the sector one may consider from how many arithmetic marks the entire
sine is made. Now part of such a given sector having been supposed, that is the
sector $ABE$, when inscribed triangle $ABE$ and circumscribed quadrilateral $ABEC$
repeat however many times, as often as the given sector is a multiple of the 
sector $ABE$ agreeing in every arithmetic mark as the square root of the whole
sine contains. Indeed this can easily be shown from consideration of the table of
Proposition 29. Precision is not required in this process, for if large radii are
used then it makes no difference if the difference be off by a few parts. The
radius $BA$ is given from the known sector $ABE$, by which the arc $AE$ is
likewise known. Let its sine $AD$ be given by $z$. Thus from the given sine
and radius, the inscribed triangle $ABE$ is given as in the previous Proposition,
as well as the circumscribed quadrilateral $ABEC$. And so the sector itself is
given, it is the second of the two continuously proportional arithmetic means of
the inscribed triangle and the circumscribed quadrilateral. Thus is given the
equation between the double of the quadrilateral $ABEC$ plus triangle $ABE$ 
and the triple of the known sector $ABE$, from whose resolution the value of
the unknown magnitude $z$ is obvious, which is sine $AD$. And by the given arc
$AE$ and its sine likewise the sine of all of the repetitions of the arc $AE$
is given from the common doctrine of the angle of a sector. Therefore the sine
of the arc initially proposed cannot hide, when it be in a given multiple of
the arc $AE$, which we wanted to find.

\begin{samepage}
\begin{center}
\large\textsc{Proposition XXXII. Problem.}

\vskip 1.0em
\emph{To find a square equal to the hyperbolic area bound by a hyperbolic
curve, one asymptote, and two a lines parallel to the other asymptote, which
space is equal to the hyperbolic sector having for its base the same curve.}
\end{center}
\end{samepage}

\begin{figure}[h!]
\includegraphics[scale=0.95]{vera_hyp_ii}
\end{figure}

Let $DIL$ be a hyperbola with asymptotes $AO$ and $AK$ that meet at the right
angle $OAK$. Consider the hyperbolic area $ILMK$ bounded by the hyperbolic
curve $IL$, the asymptotic segment $KM$, and the two lines $AK$ and $LM$, which
are parallel to the other asymptote $AO$. Choose the line $IK$ to be $10^{12}$,
$LM$ to be $10^{13}$, and $AM$ to be $10^{12}$ so that the line $KM$ is
$9\cdot 10^{12}$. We want to find the measure of the area $ILMK$. Let the lines
$IK$ and $OL$ be extended and draw the line $IP$ in order to complete the
rectangles $LNKM$ and $QIKM$. It is clear that rectangle $LNKM$ has area
$9\cdot 10^{25}$ and $QIKM$ has area $9\cdot 10^{24}$, and that the area of
quadrilateral $LIKM$ is the arithmetic mean of these rectangles, being
$4.95\cdot 10^{25}$. The geometric mean between $LNKM$ and $QIKM$ is found to
be \seqsplit{28460498941515413987990042} which is the regular circumscribed
pentagon of the hyperbolic area $LIKM$. Now as the quadrilateral $LIKM$ is to this
circumscribed pentagon, so the double of the pentagon is to the regular
inscribed hexagon of the hyperbolic area $LIKM$, namely
\seqsplit{20779754131836628160009835}. This gives the complex of the regular
hexagon with the aforementioned pentagon, of which the two areas bring about the
first terms of the convergent series. Between this is the geometric mean by
which the double of the square is divided by the same geometric mean plus the greater
term, or the circumscribed pentagon. And they give the geometric mean and bearing
whatever proportion of the second convergent terms. Thus this convergent series
of complex polygons may be continued, while the first midpoint of the terms is the
same in both convergent terms, namely up to the twentieth term, where the
circumscribed polygon is \seqsplit{2302585092993120329958961534173864} and the
inscribed is \seqsplit{23025850929931203593181124}. From here the approximation
used is that proved in Propositions 23 and 24, and the inscribed terms are
discovered, describing the true hyperbolic area of $LIKM$, being bound below by
\seqsplit{23025850929940456240178681} and the same area to be bound above by
\seqsplit{23025850929940456240178704}. And so the area can no longer hide, which
was to be shown. I thus deliver the entire series of polygons plus the number of
lines subtending the hyperbolic curve in whichever circumscribed polygon.

\begin{landscape}
\begin{center}
{\renewcommand{\arraystretch}{0.8}
\begin{tabular}{ r p{190pt} p{180pt} }
        & Circumscribed Area         & Inscribed Area \\
2       & 28460498941515413987990042 & 20779754131836628160009835 \\
4       & 24318761696971474416609403 & 22410399968461612921314879 \\
8       & 23345088913234727934949897 & 22868197570682058351436953 \\
16      & 23105412906351426185065096 & 22986193244865462241217428 \\
32      & 23045725982658962868047234 & 23015921117139340153267671 \\
64      & 23030818728479610745741910 & 23023367512879647736902891 \\
128     & 23027092819292183214705676 & 23025230015404383009313933 \\
256     & 23026161398510805910921810 & 23025695697539046352276636 \\
512     & 23025928546847571901068394 & 23025812121604634087915779 \\
1024    & 23025870334152518169052273 & 23025841227841783762272302 \\
2048    & 23025855780992551911165543 & 23025848504414868310197241 \\
4096    & 23025852142703422669729927 & 23025850323559001769499206 \\
8192    & 23025851233131194254554390 & 23025850778345089029496888 \\
16384   & 23025851005738140519209367 & 23025850892041614212944994 \\
32768   & 23025850948889877295901163 & 23025850920465745719335070 \\
65536   & 23025850934677811503232115 & 23025850927571778609090592 \\
131072  & 23025850931124795055887228 & 23025850929348286832351848 \\
262144  & 23025850930236540944102405 & 23025850929792413888218560 \\
524288  & 23025850930014477416159412 & 23025850929903445652188450 \\
1048576 & 23025850929958961534173864 & 23025850929931203593181124 \\
\end{tabular}
}
\end{center}

\vskip 0.5em
\begin{center}
The hyperbolic sector consists of the following terms 

23025850929940456240178681 \qquad\qquad 23025850929940456240178704
\end{center}
\end{landscape}

It is therefore possible without danger of error to assume the following number
for the sector of the hyperbola, of which the multiples of the number up to ten,
facilitating division thanks to the composition of the logarithm, I reveal this.
For in fact in long division it is better to use repeated subtraction for
repetition of division than ordinary division, as agrees the expert of
arithmetic.

It is clear that this problem can be resolved by the same method even if the
asymptotes $AO$ and $AK$ are not at a right angle. However we assumed
so that the problem would be made easier and more readily used in the doctrine
of logarithms, which was first discovered by our most noble Napier, and which we
have now elevated (unless I am mistaken) to the highest peak of perfection.

\begin{center}
\begin{tabular}{ r c }
1  & 23025850929940456240178700 \\
2  & 46051701859880912480357400 \\
3  & 69077552789821368720536100 \\
4  & 92103403719761824960714800 \\
5  & 115129254649702281200893500 \\
6  & 138155105579642737441072200 \\
7  & 161180956509583193681250900 \\
8  & 184206807439523649921429600 \\
9  & 207232658369454106161608300 \\
10 & 230258509299404562401787000 \\ [1.0em]
\end{tabular}
\end{center}


\begin{samepage}
\begin{center}
\large\textsc{Proposition XXXIII. Problem.}

\vskip 1.0em
\emph{To find the logarithm of any given number.}
\end{center}
\end{samepage}

\vskip 1.0em
By the same assumptions as in the preceding proposition, it is clear that,
taking $IK$ to be unity, $ML$ is ten. Therefore, $IK$ being unity, let $GH$ be
any parallel to the asymptote $AO$, the logarithm is desiered of this proposed
number. It is clear from the given line $GH$ to get $KF$, and from the preceding
to get likewise the hyperbolic area $GIKH$, which hyperbolic area I claim is
the logarithm of the proposed number $GH$. I take by the area $LIKM$ the
logarithm of the number ten. Indeed (from Gregorie of St. Vincent) the area
$GHKI$ is in the same ratio to the area $LMKI$, in which ratio $GH$ to $IK$ is
a multiple of the ratio $LM$ to $IK$. However the ratio $GH$ to $IK$ is a
multiple of the ratio $LM$ to $IK$ in the same ratio as the number $GH$ is a
multiple of the number $LM$, since it is contained itself in both ratios. Thus
the area $GIKH$ is in the same ratio to the area $LIKM$, in which the number
$GH$ is a multiple of the number $LM$, and so (since by hypothesis the area
$LIKM$ is the logarithm of the number $LM$, or ten) the area $GIKH$ shall be
the logarithm of the proposed number $GH$, since this is the essential property
of logarithms, as they be among themselves in the same direct ratio, in which
they are one another multiplied by the same number. And the logarithm of ten is
generally given as a one with some arbitrary number of zeros. If this be done,
the area $LIKM$ is to the area $GIKH$ as the arbitrary logarith of ten is to
the other number. That number shall be found to be the logarithm of the proposed
number $GH$, which we wanted to find.

\begin{center}
\large\textsc{Scholium.}
\end{center}

\vskip 1.0em
The exercise of the preceding problem set is long and laborious. Thus in order
to abbreviate our labor when composing tables of logarithms, it has been
understood that we merely need to work on the discovery of logarithms of prime
numbers. Indeed the logarithms of composite numbers are found without effort
from the primes by addition and subtraction. However as the logarithms of prime
numbers may be easily found, the order progressing from the priors to the
latters, so that from the arbitrary logarithm of 10 to each prime number 2, and
from 10 and 2 to 3, likewise from 10, 2, and 3 to 7, likewise from 10, 2, 3, and
7 to 11, and thus hereafter. Next two composite numbers differing by very little
are found, of which one is composed from a number having a known logarithm, and
so having the given logarithm, the other number is composed from only a prime
number (of which the logarithm is found) or from that plus another number having
known logarithm. Now these composite numbers are drawn (which may be, e.g., $GH$
and $EF$) to the hyperbola as parallels to the asymptote $OA$, and the
hyperbolic area $EGHF$ is found according to Proposition 32, which is done
quickly from $GH$ and $EF$, which differ by very little. By assumption, the
logarithm of one of the numbers, e.g. $GH$, is given, and so the ratio of its
logarithm to the arbitrary logarithm of ten is given, which is the same (from
the proofs to this point) as the ratio of the hyperbolic area $GIKH$ to the
hyperbolic area $LIKM$. However the area LIKM is given by Proposition 32, and
so the area $IKHG$ is known, and with the given area $EGHF$ $EIKF$ is given.
Thus the logarithm of the composite number $EF$ is given. And when by assumption
the logarithms of every number composing the number $EF$ may be given, except
that prime number of which the logarithm is desired, that logarithm of the prime
number will be given, which we wanted to find. For example, Let it be proposed
to find the logarithm of the number two, supposing arbitrarily the logarithm of
the number ten, but given as one with 25 zeros, the two composite numbers,
differing by very little, are 1000 and 1024. The logarithm of the number 1000,
or the triple of the area found above as \seqsplit{23025850929940456240178700},
namely that area given by the arbitrary logarithm of the number ten.

\begin{landscape}
\begin{center}
{\renewcommand{\arraystretch}{0.8}
\begin{tabular}{ r p{190pt} p{180pt} }
        & Circumscribed Area         & Inscribed Area \\
2       & 237170824512628449899917 & 237162487062045867846886 \\
4       & 237166655750699903737556 & 237164571388o54419219371 \\
8       & 237165613567087322970403 & 237165092476425954356426 \\
16      & 237165353021613523599438 & 237165222748948181485250 \\
32      & 237165287885271907848389 & 237165255317105572320456 \\
64      & 237165271601188181041012 & 237165263459146597159038 \\
\end{tabular}
}
\end{center}

\vskip 0.5em
\begin{center}
The hyperbolic sector consists of the following terms 

237165266173160272103220 \qquad\qquad 237165266173160458453029
\end{center}

\vskip 0.5em
Let be between these terms the four greatest of the continuously arithmetic
proportionals \seqsplit{237165266173160421183067}, which hence shall be the true
sector of the hyperbola in the proposed number of the noted, since the first
third of the noted is the same in both of the convergent terms.
\end{landscape}

\noindent The logarithm of the number 1024 is unknown, indeed is composed from
only the prime number 2, namely it is multiplied by ten. These composite numbers
are drawn to the hyperbola, as has been said, letting $GH$ be 1000 and $EF$ be
1024. But since $IK$ is \seqsplit{1000000000000}, $GH$ shall be
\seqsplit{1000000000000000} and $EF$ \seqsplit{1024000000000000}, and by
Proposition 32 the area $EGHF$ is found to be
\seqsplit{237165266173160421183067} (I give this convergent series for the
profit of the reader), or the logarithm of the number $1\frac{24}{1000}$ by the
proposed arbitrary logarithm of ten \seqsplit{23025850929940456240178700}. Next,
by the same assumed arbitrary logarithm of ten, the logarithm of the number 1000
is added, or the triple of the logarithm of ten, to the logarithm of the number
$1\frac{24}{1000}$, and will give the logarithm of the sum of the number 1024,
of which a tenth part will be the logarithm of the number two through the same
arbitrary logarithm of ten, or \seqsplit{6931471805599452914171917}. So it will
be that the logarithm of the number ten \seqsplit{23025850929940456240178700} is
to the logarithm of the number two corresponding to
\seqsplit{6931471805599452914171917} as the proposed arbitrary logarithm of the
number ten, or \seqsplit{10000000000000000000000000} is to the logarithm of the
sought number two \seqsplit{3010299956639811952405804}, which we wanted to
find\footnote{That is, $\log_10 \left(2\right) = \frac{\log \left(2\right)}{\log
\left(10\right)}$}.
By the same method the logarithm of three is found to be
\seqsplit{4771212547196624373502993}, etc.

In order to show these composite numbers, differing very little among
themselves, through one of the prime numbers, I present this table for each
prime number up to 100, as well as one rule for prime numbers between 100 and
1000 and another for prime numbers above 1000, which have all been contrived so
that the true logarithm of any prime number can be found by the corresponding
arbitrary logarithm of ten \seqsplit{10000000000000000000000000} by only one
multiplication, two divisions, and one square root extraction, as well as some
little effort.

\begin{longtable}{ c l }
\multirow{2}[2]{*}{\Huge 2}  & $1000 = 10^{3}$ \\
                          & $1024 = 2^{10}$ \\ [1.0em]
\multirow{2}[2]{*}{\Huge 3}  & $32805 = 5\cdot 6561 = 5\cdot 3^{8}$ \\
                          & $32768 = 2^{15}$ \\ [1.0em]
\multirow{2}[2]{*}{\Huge 7}  & $2400 = 3\cdot 32 = 3\cdot 2^{5}$ \\
                          & $2401 = 7^{4}$ \\ [1.0em]
\multirow{2}[2]{*}{\Huge 11} & $9800 = 2\cdot 49\cdot 100 = 2\cdot 7^{2}\cdot 10^{2}$ \\
                          & $9801 = 121\cdot 81 = 11^{2}\cdot 3^{4}$ \\ [1.0em]
\multirow{2}[2]{*}{\Huge 13} & $123200 = 7\cdot 11\cdot 25\cdot 64 = 7\cdot 11\cdot 5^{2}\cdot 2^{6}$ \\
                          & $123201 = 169\cdot 729 = 13^{2}\cdot 3^{6}$ \\ [1.0em]
\multirow{2}[2]{*}{\Huge 17} & $2600 = 13\cdot 8\cdot 25 = 13\cdot 2^{3}\cdot 5^{2}$ \\
                          & $2601 = 9\cdot 289 = 3^{2}\cdot 17^{2}$ \\ [1.0em]
\multirow{2}[2]{*}{\Huge 19} & $28899 = 169\cdot 9\cdot 19 = 13^{2}\cdot 3^{2}\cdot 19$ \\
                          & $28900 = 100\cdot 289 = 10^{2}\cdot 17^{2}$ \\ [1.0em]
\multirow{2}[2]{*}{\Huge 23} & $25920 = 10\cdot 32\cdot 81 = 10\cdot 2^{5}\cdot 3^{2}$ \\
                          & $25921 = 49\cdot 529 = 7^{2}\cdot 23^{2}$ \\ [1.0em]
\multirow{2}[2]{*}{\Huge 29} & $613088 = 17\cdot 23\cdot 32\cdot 49 = 17\cdot 23\cdot 2^{5}\cdot 7^{2}$ \\
                          & $613089 = 729\cdot 841 = 3^{6}\cdot 29^{2}$ \\ [1.0em]
\multirow{2}[2]{*}{\Huge 31} & $116280 = 10\cdot 17\cdot 19\cdot 4\cdot 9 = 10\cdot 17\cdot 19\cdot 2^{2}\cdot 3^{2}$ \\
                          & $116281 = 121\cdot 961 = 11^{2}\cdot 31^{2}$ \\ [1.0em]
\multirow{2}[2]{*}{\Huge 37} & $165648 = 3\cdot 7\cdot 17\cdot 29\cdot 16 = 3\cdot 7\cdot 17\cdot 29\cdot 2^{4}$ \\
                          & $165649 = 121\cdot 1369 = 11^{2}\cdot 37^{2}$ \\ [1.0em]
\multirow{2}[2]{*}{\Huge 41} & $1413720 = 7\cdot 10\cdot 11\cdot 17\cdot 4\cdot 27 = 7\cdot 10\cdot 11\cdot 17\cdot 2^{2}\cdot 3^{3}$ \\
                          & $1413721 = 1681\cdot 841 = 41^{2}\cdot 29^{2}$ \\ [1.0em]
\multirow{2}[2]{*}{\Huge 43} & $978120 = 10\cdot 11\cdot 13\cdot 19\cdot 4\cdot 9 = 10\cdot 11\cdot 13\cdot 19\cdot 2^{2}\cdot 3^{2}$ \\
                          & $978121 = 529\cdot 1849 = 23^{2}\cdot 43^{2}$ \\ [1.0em]
\multirow{2}[2]{*}{\Huge 53} & $664848 = 7\cdot 19\cdot 23\cdot 8\cdot 125 = 7\cdot 19\cdot 23\cdot 2^{3}\cdot 5^{3}$ \\
                          & $664849 = 9\cdot 121\cdot 2809 = 3^{2}\cdot 11^{2}\cdot 53^{2}$ \\ [1.0em]
\multirow{2}[2]{*}{\Huge 57} & $5851560 = 3\cdot 5\cdot 13\cdot 31\cdot 121 = 3\cdot 5\cdot 13\cdot 31\cdot 11^{2}$ \\
                          & $5851561 = 1681\cdot 3481 = 41^{2}\cdot 57^{2}$ \\ [1.0em]
\multirow{2}[2]{*}{\Huge 61} & $3575880 = 5\cdot 7\cdot 11\cdot 43\cdot 8\cdot 27\cdot = 5\cdot 7\cdot 11\cdot 43\cdot 2^{3}\cdot 3^{3}$ \\
                          & $3575881 = 961\cdot 3721 = 31^{2}\cdot 61^{2}$ \\ [1.0em]
\multirow{2}[2]{*}{\Huge 67} & $1620528 = 3\cdot 13\cdot 16\cdot 49 = 3\cdot 13\cdot 2^{4}\cdot 7^{2}$ \\
                          & $1620529 = 361\cdot 4489 = 19^{2}\cdot 67^{2}$ \\ [1.0em]
\multirow{2}[2]{*}{\Huge 71} & $2016399 = 3\cdot 11\cdot 29\cdot 43\cdot 49 = 3\cdot 11\cdot 29\cdot 43\cdot 7^{2}$ \\
                          & $2016400 = 16\cdot 25\cdot 5041 = 2^{4}\cdot 5^{2}\cdot 71^{2}$ \\ [1.0em]
\multirow{2}[2]{*}{\Huge 73} & $5116644 = 4\cdot 9\cdot 169\cdot 841 = 2^{2}\cdot 3^{2}\cdot 13^{2}\cdot 29^{2}$ \\
                          & $5116645 = 7\cdot 17\cdot 19\cdot 31\cdot 73$ \\ [1.0em]
\multirow{2}[2]{*}{\Huge 79} & $5997600 = 17\cdot 32\cdot 9\cdot 25\cdot 49 = 17\cdot 2^{5}\cdot 3^{2}\cdot 5^{2}\cdot 7^{2}$ \\
                          & $5997601 = 961\cdot 6241 = 31^{2}\cdot 79^{2}$ \\ [1.0em]
\multirow{2}[2]{*}{\Huge 83} & $1164240 = 5\cdot 11\cdot 16\cdot 27\cdot 49 = 5\cdot 11\cdot 2^{4}\cdot 3^{3}\cdot 7^{2}$ \\
                          & $1164241 = 169\cdot 6889 = 13^{2}\cdot 83^{2}$ \\ [1.0em]
\multirow{2}[2]{*}{\Huge 89} & $2859480 = 5\cdot 47\cdot 8\cdot 9\cdot 169\cdot = 5\cdot 47\cdot 2^{3}\cdot 3^{2}\cdot 13^{2}$ \\
                          & $2859481 = 361\cdot 7921 = 19^{2}\cdot 89^{2}$ \\ [1.0em]
\multirow{2}[2]{*}{\Huge 97} & $1138488 = 3\cdot 13\cdot 41\cdot 89\cdot 8 = 3\cdot 13\cdot 41\cdot 89\cdot 2^{3}$ \\
                          & $1138489 = 121\cdot 9409 = 11^{2}\cdot 97^{2}$ \\ [1.0em]
\end{longtable}

For prime numbers between 100 and 1000 let this be the rule: before the prime
number of which the logarithm is desired, the two numbers immediately preceding
are assumed, and the number following immediately after it, which three numbers
with that prime are four numbers following one after another in natural order
among themselves. Next the first number is multiplied by the cube of the third
and the fourth by the cube of the second, and it will be that their difference
equals the sum of the prime and the fourth or of the second and the third, as
can easily be shown. These numbers have at least six prime factors between them,
and thus they differ very little among themselves. Also the logarithms of the
four of these numbers (except the third) are known from the preceding method,
and thus are suitable to our abbreviation. So much apparatus is not useful in
numbers beyond 1000, since the rectangle of the numbers, among which the prime
number is understood immediately of which the logarithm is desired, which is
only less the square of the prime number by one. And so these have at least six
prime factors among them, and the logarithms of the first and third are
obtained. Therefore the infinitude are available to us.

\begin{samepage}
\begin{center}
\large\textsc{Proposition XXXIV. Problem.}

\vskip 1.0em
\emph{From a given logarithm to find its number.}
\end{center}
\end{samepage}

\vskip 1.0em
From the demonstration it is clear that this same problem be as if that was
proposing. From the given hyperbolic area, and one line understood to be
parallel to one of the asymptotes, to find another area and its parallel to
the asymptote. It may be considered from however many arithmetic terms the
arbitrary logarithm of ten is comprised, and some part of the logarithm, or
the given area, is assumed, namely the area $LIKM$, so that the regular
circumscribed polygons of the area $LIKM$ and the regular inscribed hexagons of
the same be repeated however many times, as often as is repeated the given area
to the area $LIKM$, agreeing in all arithmetic terms, as many the square root
of the arbitrary logarithm of ten contains. Indeed this can be done easily from
the table of Proposition 32. Therefore the measure of the area $LIKM$ is
obtained and the line $IK$ is unity by assumption. Let $LM$ be $z$. As in
Proposition 32 the regular circumscribed pentagon and the regular inscribed
hexagon give the area $LIKM$, between which the given area $LIKM$ is the
second of the two continuously proportional arithmetic means. And so the double
of the hexagon plus the pentagon is equal to the triple of the area, of which
equation the unknown $z$, or the number $LM$, clearly resolves, of which
repeated however many times, as many as the area $LIKM$ is submultiplied to the
space, or the given logarithm, is the desired number, which we wanted to find.

This is the same problem as Proposition 8, but more general and the method of
this solution is far less work.

\begin{samepage}
\begin{center}
\large\textsc{Proposition XXXV. Problem.}

\vskip 1.0em
\emph{A line having been drawn through a given point on a diamter, to divide
the semicircle into a given ratio.}
\end{center}
\end{samepage}

\begin{figure}[h]
\includegraphics[scale=0.95]{vera_semicirc}
\end{figure}

\vskip 1.0em
Let $ADG$ be a semicircle of given diameter $AG$, center $E$, and $B$ be a given
point on the diameter. Assume it is done as desired, and let the line $BD$
divide the semicircle into a given ratio. Since the measure of the semicircle is
given and the ratio into which it is divided, so its portion, $DBG$, is given.
Let $z$ be the line $BD$. From the given lines $BD$, $BE$, and $ED$, the
triangles $DEB$, $DEF$, and $DEG$ are made known. Next let it be that as $DEF$
plus $DEG$ is to $DEG$ so the double of $DEG$ is to the circumscribed
quadrilateral $DEGH$. Setting $DEG$ and $DEGH$ as the first convergent term, the
convergent series of complex ploygons may be continued, repeated as often as
necessary according to the properties of the circle. Until the agreed upon
approximation to employ such that the sector $DEG$ is shown, which plus the
triangle $DBE$ is euqal to the known portion $DBG$, the equation of which
clearly resolved the unknown magnitude $z$, or the line $BD$. The rest is
obvious.

The same problem is resolved by altogether the same method in the ellipse, the
hyperbola, or any sector given.

\begin{center}
\large\textsc{Scholium.}
\end{center}

\vskip 1.0em
If a problem set of the previous Propositions is desired for mechanical pratice,
it will not be difficult to imitate the calculation, approximation, and 
resolution of equations to some extent according to the common rules of the
practice of Geometry. Many such problem sets may be resolved by the power of
analysis and by our rules of convergent series, which before may have been
impossible to estimate. However, it will be strongly said by anyone that these
solutions are not geometric. I respond that if the only practice understood by
the geometer is the power of the straightedge and compass, not only will this be
impossible but likewise will every problem set which cannot be reduced to a
quadratic equation, as may easily be shown. And if the reduction of the problem
to an analytic equation be understood by the geometer, all of this problem set
are impossible to the geometer, where by this proof it is clear that such a
reduction is cannot be done. If in truth this most simple method of every
possibility be understood by the geometer, it will be found most strongly after
timely consideration that the entirety of the above problem set may be resolved
most geometrically. Carefully observing the whole doctrine of convergent series
it is possible likewise by little effort to apply it to simple series. Indeed
let $A$, $B$, $C$, $D$, $E$, etc. be a series of such nature that the third term
$C$ is composed by the same method from the first and second terms $A$ and $B$
as the fourth term $D$ is composed from the second and third terms $B$ and $C$,
and the fifth $E$ from the third and fourth $C$ and $D$, and so on infinitely.
Let also the difference of the aforementioned $A$ and$B$ be always greater than
the difference of the subsequent terms $B$ and $C$. We may assume this series to
continue infinitely until two of the adjacent terms are not different, and
letting one of these terms be $z$, which we call the limit of the series. I
claim that $z$ is composed by the same method from $A$ and $B$ as from $B$ and
$C$ or $C$ and $D$. The proof scarcely differes from that of Proposition 10 and
its Conclusions. If this ratio is put to a triangle, inscribed in a sector of a
circle or ellipse or circumscribed to a sector of a hyperbola, $a$, and a
quadrilateral, regularly inscribed in a circle or ellipse or circumscribed to a
hyperbola, $b$, then the hexagon regularly inscribed in a sector of a circle or
ellipse or circumscribed to a hyperbola will be $\sqrt{\frac{2b^{3}}{a+b}}$.
Thus the sector of a circle, ellipse, or hyperbola is composed of the same method
from $a$ and $b$ as from $b$ and $\sqrt{\frac{2b^{3}}{a+b}}$. And so this
likewise can be shown, that the ratio of the sector to its given triangle may not be
analytic, according to Proposition 11. It would actually be possible to yet
prove by another particular method that the circular arc does not have an
analytic ratio to its given chord, but I do not add more, meanwhile advising
geometers to growth by science. I myself might have discovered in certain
figures (which Descartes called the second type) three foci, or three points,
from which lines drawn to any point of the curve the sum or difference is always
the same. Whence it appears to me to be true as all curves of the first type
have two foci whether real or imaginary, as all of the second type have three,
all third four, and so on infinitely. This speculation is certainly most worthy
of scrutiny, and indeed it may be an extraordinary property of the geometric
figures, and of the most useful mechanical practice of all equations.

\begin{center}
\Huge FINIS.

\vskip 0.5em
Anno Dom: 1667.
\end{center}

\cleardoublepage
\appendix
\begin{center}
\LARGE\textsc{Appendix.} \\
\large\emph{A modern rendering of the propositions.}
\end{center}

\vskip 2.0em

\begin{center}
Proposition I:
\end{center}
Let the area of $BAPF=a$, the area of $BAP=b$, and the area of $BAPI = c$. Then
$c = \frac{a + b}{2} = \textrm{AM} \left( a, b \right)$.

\vskip 2.0em
\begin{center}
Proposition II:
\end{center}
Let the area of $AFBP = a$, the area of $ABIP = b$, and the area of $ABDLP = c$.
Then $c = \frac{2ab}{a+b} = \frac{\textrm{GM} \left( a, b \right)}{\textrm{AM}
\left( a, b \right)}$.

\vskip 2.0em
\begin{center}
Proposition III:
\end{center}
Let the area of $ABIP = a$, the area of $BAP = b$, and the area of $ABDLP = c$.
Then $c = \frac{2a^2}{a+b} = \frac{a^2}{\textrm{AM} \left( a, b \right)}$.

\vskip 2.0em
\begin{center}
Proposition IV:
\end{center}
Let the area of $ABIP = a$, the area of $ABDLP = b$, and the area of $ABEIOP =
c$. Then $c = \frac{a + b}{2} = \textrm{AM} \left( a, b \right)$.

\vskip 2.0em
\begin{center}
Proposition V:
\end{center}
Let the area of $ABEIOP = a$, the area of $ABIP = b$, and the area of polygon
$ABCGKNP = c$. Then $c = \frac{2a^2}{a+b} = \frac{a^2}{\textrm{AM} \left( a, b
\right)}$.

\vskip 2.0em
\begin{center}
Proposition VI:
\end{center}
Let the area $ABP = a$, the area of $ABFP = b$, the area of $ABIP = c$, and the
area of $ABDLP = d$. Then $a - b \leq 2\left( c - d \right) = \frac{(a + b)^2}{a
+ 3b} = \frac{2\textrm{AM}^2 \left( a, b \right)}{\textrm{AM} \left( a, b
\right) + b}$.

\vskip 2.0em
\begin{center}
Proposition VII:
\end{center}
Let $a_{0} = a$ and $b_{0} = b$ where $a < b$ and 


\bibliographystyle{plain}
\bibliography{cite1}
\end{document}















